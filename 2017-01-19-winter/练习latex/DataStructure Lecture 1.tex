\documentclass{beamer}
\usepackage{xeCJK}

\usetheme{PaloAlto}
\title{数据结构 第一讲}
\subtitle{树状数组、线段树、及一些非正规数据结构}
\author{作者}
\date{\today}

\begin{document}
	\maketitle
	\begin{frame}{目录}
		\tableofcontents
	\end{frame}
	
	\section{树状数组}
	\begin{frame}
		\frametitle[长标题]{这是一个比较比较长的标题}
		这是简单的一帧
		
		据说是垂直显示。
	\end{frame}
	
	\section{线段树}
	\begin{frame}
		\frametitle{另一页}
		这没有什么内容。
	\end{frame}
	
	\section{最大公约数}
	\begin{frame}{最大公约数}
		\begin{block}{块标题}
			块里面的内容
			
			拉拉
		\end{block}
		\begin{definition}[最大公约数]
			数a,b的最大公约数是指其公约数中最大的那一个。
		\end{definition}
		\begin{theorem}[最大公约数]
			d 是 a, b的最大公约数当且仅当它是a,b的约数中最大的一个。
		\end{theorem}
		\begin{proof}
			当然显然啦。
		\end{proof}
	\end{frame}
\end{document}