\documentclass{article}
\usepackage{xeCJK}
\usepackage{ulem}         		%  波浪线 和 叉叉
\usepackage{setspace}  		% 设置行间距
\usepackage{paralist} 			% 压缩列表
\usepackage{amsmath}		% 数学
\usepackage{fancyvrb}
\author{丁尧尧}
\date{\today}
\title{学习\LaTeX 的基本命令}
\begin{document}
	\maketitle
	\begin{abstract}
		这里是用来练习\LaTeX 命令的一个文档。用于备忘和查阅。
	\end{abstract}
	\setcounter{tocdepth}{2}	
	\tableofcontents
	\part{基本部分}
	第一部分的介绍
		\section{文字}
		主要是文本模式(区别于数学模式)下的文本格式。
			\subsection{字符的输入}
			需要转义的普通字符			\\ 			
			\# 			\\ 
			\$  		\\ 
			\^ 			\\ 
			\& 			\\ 
			\_ 			\\ 
			\{ 			\\ 
			\} 			\\ 
			\~ 			\\
			\textbackslash			\\
			\%			\\
			不同的横线个数表达含义不同,分别是英文连字符、数字连字符、扩折号。 \\
			-			\\
			--			\\
			---			\\
			特殊字符:			\\
			\textcopyright			\\
			\textregistered			\\
			\today			\\
			\={A}			\\
			\~{A}			\\
			\^{A}
			\subsection{字体样式和大小}
			\textbf{Blod}			\\
			\textit{Italic}			\\
			\textrm{Roman}			\\
			\textsf{Sans serif}			\\
			\texttt{Monospace}			\\
			\tiny tiny \\
			\scriptsize scriptsize \\
			\footnotesize footnotesize \\
			\small small \\
			\normalsize normalsize \\
			\large large \\
			\Large Large \\
			\LARGE LARGE \\
			\huge huge \\
			\Huge Huge \\
			\normalsize  
			\emph{emphasis}  \\
			\underline{underline}	\\
			\uwave{uwave} \\
			\sout{sout}  \\
			\textbackslash newline {} 或 {} \textbackslash \textbackslash 换行			\\
			\textbackslash newpage {}换页 \\
		\section{段落相关}
			\subsection{段落对齐}
			\begin{flushleft}
				这是左对齐  \\
				再来一行看看
			\end{flushleft}
			\begin{flushright}
				这是右对齐  \\
				再来一行看看
			\end{flushright}
			\begin{center}
				居中对齐				\\
				再来一行看看
			\end{center}
			\subsection{行间距}
			\begin{spacing}{2.0}
			第一行\\第二行\\第三行\\
			\end{spacing}
			\subsection{引用}
			\begin{quote} 
			    引用 \\两端都缩进
		    \end{quote}
		    \begin{quotation}
		    	引用 \\ 和上面差不多,但多了首行缩进
		    \end{quotation}
		    \subsection{原文打印}
		    \verb|printf( "hello, world!\n );|
		    \begin{verbatim}
				#include <cstdio>
				    int main() {
				    printf( "hello,world!\n );
				    return 0;
				}
		    \end{verbatim}
		    \begin{verbatim*}
   				#include <cstdio>
	   			    int main() {
	   			    printf( "hello,world!\n );
	   			    return 0;
   				}
		    \end{verbatim*}
		    \label{CODE}
			\subsection{脚注}
			这是一段话,哈哈哈哈哈\footnote{此处的哈用于表达高兴}
		\section{列表}
			\subsection{基本列表}
			下面是三种基本列表
			\begin{itemize}
				\item C++
				\item Java
				\item HTML
			\end{itemize}
			\begin{enumerate}
				\item C++
				\item Java
				\item HTML
			\end{enumerate}
			\begin{description}
				\item[C++] 编程语言
				\item[Java] 编程语言
				\item[HTML] 标记语言
			\end{description}
			还可以改变前面的符号
			\renewcommand{\labelitemi}{-}
			\renewcommand{\theenumi}{\alph{enumi}}
			\begin{itemize}
				\item C++
				\item Java
				\item HTML
			\end{itemize}
			\begin{enumerate}
				\item C++
				\item Java
				\item HTML
			\end{enumerate}
		\section{交叉引用}
		代码在\pageref{CODE}页\ref{CODE}节
	\part{数学}
	数学相关
		\section{数学模式}
		插入公式可以在一行中直接插入,比如$F=ma$,也可以单独起一行,无标号的:
		\[ F=ma \]
		有标号的:
		\begin{equation}
			F=ma
		\end{equation}
		有的时候强调某一部分,可以给它加个方框:
		\[  F=m\boxed{a} \]
		\section{基本元素}
		英文字母可以直接输入,拉丁文有对应的符号:\\
		\[
		\alpha\;\beta\;\gamma\;\delta\;\epsilon\;\theta\;\lambda\;\mu\;
		\mu\;\eta\;\xi\;\pi\;\rho\;\sigma\;\varphi\;\Gamma\;\Delta\;\Theta\;
		\Lambda\;\Xi\;\Pi\;\Sigma\;
		\]
		上下标:
		\[
		x_{i,j}^{2} \quad \sqrt{x} \quad \sqrt[n]{x}
		\]
		分数:
		\[
		\frac{a^3}{\sqrt{b}}
		\]
		小运算符:
		\[
		+ \; - \; * \; / \; = \; \pm \; \times \; \div\;  \cdot\; \cap\; \cup\; \geq\;  \leq\; \neq\;
		\approx\; \equiv\;
		\]
		大运算符: \\
		$\sum\limits_{i=1}^{n} i \quad \lim\limits_{x\to 0} x^2 \quad \int \prod $
		\[
		\int_{a}^{b}\frac{1}{x}dx = ln(b) - ln(a)
		\]
		箭头:\\
		\[
		\leftarrow\;\rightarrow\;\leftrightarrow\;\Leftarrow\;\Rightarrow\;\Leftrightarrow\;
		\longleftarrow\;\longrightarrow\;\longleftrightarrow\;\Longleftarrow\;\Longrightarrow\;\Longleftrightarrow\;
		\xleftarrow{x+y+z=r}
		\]
		标注:\\
		\[
		\bar{x} \; \vec{x} \; \hat{x} \; \check{x}\;
		\]
		各种括号:\\
		\[ 	\Bigg ( \bigg ( \Big ( \big ( (x) \big ) \Big ) \bigg ) \Bigg )  \]  
		\[ [x]  \]
 		\[ \langle x\rangle \]
		\[ \lvert x \rvert \]
		\[ \lVert x \rVert \]
		省略号:\\
		\[
			x_1, x_2, \dots , x_n \quad 1 , 2, \cdots , n \quad \vdots \quad \ddots 
		\]
		间距:\\
		\[
		(\!) \quad (\,) \quad (\:) \quad (\;) \quad (\quad) \quad (\qquad) 
		\]
		\section{矩阵}
		矩阵:\\
		\[
		\begin{array}{cccc}
			x_{1,1} & x_{1,2} & \cdots & x_{1,n} \\
			x_{2,1} & x_{2,2} & \cdots & x_{2,n} \\
			\vdots & \vdots & \ddots & \vdots \\
			x_{n,1} & x_{n,2} & \cdots & x_{n,n} \\
		\end{array} \quad
		\begin{pmatrix} a & b \\ c & d \end{pmatrix} \quad
		\begin{bmatrix} a & b \\ c & d \end{bmatrix} \quad
		\begin{Bmatrix} a & b \\ c & d \end{Bmatrix} \quad
		\begin{vmatrix} a & b \\ c & d \end{vmatrix} \quad
		\begin{Vmatrix} a & b \\ c & d \end{Vmatrix} 
		\]
		行间矩阵$( \begin{smallmatrix}
			a & b \\ c & d
		\end{smallmatrix} )$
		\section{多行公式}
		\section{定理和证明}
		\section{数学字体}		
\end{document}