\documentclass[9pt]{beamer}
\usepackage{xeCJK}

\usetheme{Berkeley}
\usefonttheme[onlymath]{serif}
\usepackage[utf8]{inputenc}
\usepackage[T1]{fontenc}
\usepackage{amsmath}
\usepackage{amsfonts}
\usepackage{amssymb}
\usepackage{multicol}


\author{丁尧尧}

\title{数论、组合选讲}

\institute{上海交通大学}

\date{\today}

%\setbeamercovered{transparent}

%\setbeamertemplate{navigation symbols}{}

\begin{document}
	\maketitle
	
	\begin{frame}{目录}
		\begin{multicols}{2}
			\tableofcontents
		\end{multicols}
	\end{frame}
	\section{组合数、排列数} 
	\begin{frame}{组合数、排列数} 
		从$ n $个对象中选$m$个排成一列的方案数称作排列数,记作$ P(n,m) $。
		
		如果$n=m$,则称为全排列,记作$ P(n) $。
		
		从$ n $个对象中选择$ m $个的方案数称作组合数,记作$ C(n,m) $或$\binom{n}{m}$。
		
		它们满足:\footnote{我们将$0!$看成$1$}
		
		\begin{itemize}
			\item $ P(n,m) = \frac{n!}{(n-m)!} $
			\item $ \binom{n}{m} = \frac{n!}{m!(n-m)!} $  (常用来算模意义下组合数)
			\item $ \binom{n}{m} = \binom{n-1}{m-1}\binom{n-1}{m} $ (常用来算一般意义下组合数)
			\item $(x+y)^n = \sum_{i=0}^{n}\binom{n}{i}x^iy^{n-i} $ (二项式定理)
			\item $ \sum_{i=0}^{n}\binom{n}{i} = 2^n $ (令上面$x = y = 1$)
		\end{itemize}
		
		还有一些重要的东西:
		\begin{itemize}
			\item 可重排列
			\item 循环排列
			\item 不区分球,区分盒子(夹棍法)
			\item 卡特兰数列
		\end{itemize}
	\end{frame}
	\section{容斥原理、鸽巢原理} 
		\begin{frame}{容斥原理、鸽巢原理} 
			容斥原理:
			$$
				\mid \bigcup_{i = 1}^{n} A_i\mid = \sum_{i} \mid  A_i \mid - \sum_{1 \leq i < j \leq n} \mid A_i \cap A_j \mid + \sum_{1 \leq i < j < k \leq n} \mid A_i \cap A_j \cap A_k \mid \cdots 
			$$
			鸽巢原理:将$n$个鸽子塞进$n-1$个巢中,那么必定有一个巢有至少两个鸽子。
		\end{frame}
	\section{快速幂} 
		\begin{frame}{快速幂} 
			如果我们要计算$a^n$,那么我们可以将$n$写成二进制形式,然后将$a^n$拆成一些$a^{2^i}$的乘积,而后者可以递推来算。
			
			如果我们的$a$特别大,大到64位的整型都存不下,并且是以十进制的形式输入的,那么我们可以弄十进制快速幂(类比二进制快速幂)。
			
			快速乘也是类似的思想。
		\end{frame}
	\section{最大公约数} 
		\begin{frame}{最大公约数} 
			两个整数公共的约数称为公约数,如果这两个数不同时为$ 0 $,那么他们中就存在最大的一个公约数,称为最大公约数,记作$gcd(a,b)$。
			
			两个不为$0$的整数公共的倍数称为公倍数,其中最小的正公倍数记为最小公倍数,记作$lcm(a,b)$。
			
			它们有如下性质:
			
			\begin{itemize}
				\item $ gcd(a,b) = gcd(b,a) $
				\item $ gcd(a,0) = abs(a) $
				\item $ gcd(a,b) = gcd(a,b+ax), x \in Z $	以上两条是我们求$gcd$的主要途径
				\item $ ab = gcd(a,b)lcm(a,b) $  用来求$lcm$。
			\end{itemize}
		\end{frame} 
	\section{扩展欧几里得} 
		\begin{frame}{扩展欧几里得} 
			关于$gcd(a,b)$有一个重要的事实,那就是存在整数$x,y$使得:
			$$
			gcd(a,b) = ax+by
			$$
			我们可以用辗转相除法给出构造性的证明。

			从而我们也有了求$x,y$的方法。
			
			事实上,如果我们让$x,y$遍历整个整数集合,那么$ax+by$就会遍历所有$gcd(a,b)$的倍数。
			
			有了上面事实,我们就可以证明(虽然它们感觉很显然):
			$$
				a \mid bc, \;  gcd(a,b) = 1 \Rightarrow a \mid c
			$$
			$$
				p \mid a_1a_2 \Rightarrow p \mid a_1 \; or \; p \mid a_2
			$$
			从而证明唯一分解定理。
		\end{frame}
	\section{二元一次不定方程} 
		\begin{frame}{二元一次不定方程} 
			问题:
			\quad 给定$a,b$,讨论下面这个二元一次不定方程解的情况:
			$$
				ax + by = c
			$$
			我们设$d = gcd(a,b)$。那么:

			\quad 如果 $d \nmid c$,无解
			
			\quad 如果 $d \mid c$,那么有无数解,并且解集和方程$\frac{a}{d}x+\frac{b}{d}y = \frac{c}{d}$的解集相同。这时我们可以找到$1 = gcd(\frac{a}{d},\frac{b}{d}) = \frac{a}{d}x_0'+\frac{b}{d}y_0'$的解$x_0',y_0'$,从而得到原方程的一个特解$x_0 = \frac{c}{d}x_0', y_0 = \frac{c}{d}y_0'$。整个解集就是$x = x_0 + \frac{b}{d}t, \quad	y = y_0 - \frac{a}{d}t	$,其中$t$取遍整个$Z$。
			
		\end{frame}
	\section{同余}
		\begin{frame}{同余}		
		两个数除以某个数有相同的余数是一个重要的关系,所以我们引进同余符号:
		
		$$
			a \equiv b \; ( mod \; m ) \Leftrightarrow m \mid a - b
		$$
		
		有些性质(如果后面没有模数,默认为$(mod \; m)$):
		\begin{itemize}
			\item $a \equiv b, b \equiv c \Rightarrow a \equiv c$
			\item $a \equiv b, c \equiv d \Rightarrow a + c \equiv b + d, ac \equiv bd$
			\item 对于非$0$的整数$c$,有$a \equiv b \; (mod \; m) \Leftrightarrow ac \equiv bc \; (mod \; mc) $
			\item $ac \equiv bc \; ( mod \; m ) \Leftrightarrow a \equiv b \; ( mod \;\frac{m}{gcd(m,c)}) $
		\end{itemize}
		
		\end{frame} 
	\section{欧拉函数} 
		\begin{frame}{欧拉函数} 
			我们定义欧拉函数$\varphi(n)$:
			$$
				\varphi(n) = \mid  \{ a \in Z \mid 1 \leq a \leq n \; and \; gcd(a,n) = 1 \} \mid 
			$$
			即$\varphi(n)$表示$1$到$n$中和$n$互质的数的个数。
			
			关于它,有以下事实:
			
			\begin{itemize}
				\item $\varphi(nm) = \varphi(n)\varphi(m)  \quad (gcd(m,n) = 1)$  积性函数
				\item $\varphi(n) = n\prod_{p \mid n} (1 - \frac{1}{p}) $ 用于手算
				\item $ n = \sum_{d \mid n}\varphi(d)$  
			\end{itemize}
			
		\end{frame}
		
		\begin{frame}{欧拉定理}
			关于欧拉函数,我们还有一个重要的定理:
			$$
				a^{\varphi(n)} \equiv 1 \; (mod \; m)  \;(gcd(a,m) = 1)
			$$
			这个定理的一个直接推论就是费马小定理:
			$$
				a^{p-1} \equiv 1 \; (mod \; p) \; (p \; is \;  a \; prime)
			$$
			欧拉定理的证明,大致过程是说$m$的一个缩系中每个数乘以一个与$m$互素的数后还是$m$的一个缩系,然后两个缩系中每个数乘起来同余,然后就有了欧拉定理。
			
			我们一般用欧拉定理去求逆元,或将大指数变小。
		\end{frame} 
	\section{求逆元}  
		\begin{frame}
			有这样一个问题,对于整数数$a$,求一个数$b$满足:
			$$
			ab \equiv 1 \; (mod \; m)
			$$
			我们可以证明上面这个式子成立当且仅当有:
			$$
			gcd(a,m) = 1
			$$
			并且将$b$记作$a^{-1}$,称为$a$在模$m$意义下的逆。
			
			有了这个,我们就可以在模意义下做除法操作了。
		\end{frame}
	\section{中国剩余定理} 
		\begin{frame}{中国剩余定理} 
			还有一类方程我们需要求解:
			$$
				x \equiv a_i \; (mod \; m_i) \quad ( i = 1, 2, \cdots, n )
			$$
			这是一个同余方程组,$x$的系数都是$1$,并且满足$m_i$两两互质。
			
			我们设$M = m_1m_2m_3\dots m_n$,上面那个方程就等价于:
			$$
				x \equiv \sum_{i = 1}^{n} \frac{M}{m_i} (\frac{M}{m_i})^{-1}a_i \; (mod M)
			$$
			其中$(\frac{M}{m_i})^{-1}$指的是关于$m_i$的逆元。
			
			中国剩余定理除了单纯的解方程外,还为我们提供了一种思路,就是当题目中给出的模数不是质数时,我们可以把它质因分解成$p_1^{\alpha_1}p_2^{\alpha_2}\dots p_s^{\alpha_s}$的形式,然后对每个$p_i^{\alpha_i}$做(这个时候会有一些比合数更好的性质),然后再用中国剩余定理合并起来。
		\end{frame} 
	\section{Lucas定理}	 
		\begin{frame}{Lucas定理}	 
			求组合数$\binom{n}{m}$是我们经常遇到的问题,如果是要求出其具体值,我们一般是用组合数的递推公式来直接做(因为组合数增长很快,所以规模一般很小)。
			
			如果是在摸意义下,那么就可以弄得很大,而$Lucas$定理就是用来处理模数是小素数($p \leq 10^6$),但$n,m$可以很大($n,m \leq 10^{18}$)的情况,它的定理内容是:
			$$
				\binom{n}{m} \equiv \binom{qp+r}{sp+t} \equiv \binom{q}{s}\binom{r}{t} \; ( mod \; p)
			$$
			其中
			$$
				n = qp+r, m = sp+t, 0 \leq r, t < p
			$$
			我们可以对$\binom{q}{s}$继续用定理,从而将$\binom{n}{m}$分解成一些小的数对应组合数的乘积,并且将后者排成一排,可以看出上面部分是$n$的$p$进制分解,下面是$m$的$p$进制分解。
		\end{frame}
	\section{筛素数} 
		\begin{frame}{筛素数} 
			我们怎么去把一定范围内的素数全部搞出来?
			\begin{itemize}
				\item $Eraosthenes$筛,思路是从前往后枚举每个数,每枚举到一个素数,就用它把比他大的倍数都标记为“非素数”。复杂度$O(nlog(log(n))$
				\item $Euler$筛,一个数一定是被它的最小素因子筛掉的。复杂度$O(n)$
			\end{itemize}
			其实在$10^6$范围内,两个的速度差不多(我试验了下,$10^6$时前者基本操作大概是后者的六倍,但是后者中有取模运算,所以弄得差不多快),前者比后者更显然一些,后者比前者更容易计算积性函数一些。
		\end{frame}
\end{document}