\documentclass{article}
\usepackage{xeCJK}

\usepackage{amsmath}
\usepackage{amsfonts}
\usepackage{amssymb}

\begin{document}
	\section{modeq}
	 这道题就是一个普通的解方程,只是需要你输出两个特殊的解,我们如果有方程:
	$$
		ax + by = c
	$$
	我们首先用扩展欧几里得得到下面方程的解:
	$$
		ax_0 + by_0 = d \; ( d = (a,b) )
	$$
	如果$ d \nmid c $,则无解。
	如果$ d \mid c $,则有解,在方程两边同时乘以$ \frac{c}{d} $,得到:
	$$
		a(x_0\frac{c}{d}) + b(y_0\frac{c}{d}) = c
	$$
	所有解满足:
	$$
		x = x_0\frac{c}{d} + t\frac{b}{d}
	$$
	$$
		y = y_0\frac{c}{d} - t\frac{a}{d}
	$$
	其中$t \in Z$。
	所以我们只需要将原方程得到的任一解的$x_0$对$\frac{b}{d}$取模,或$y_0$对$\frac{a}{d}$取模得到想要的解。
	
	\section{crt}
	中国剩余定理,因为$m$两两不互质,我们只能两两合并来做,首先把两个方程:
	$$
		x \equiv a_1 ( m_1 ) \quad x \equiv a_2 ( m_2 ) 
	$$
	等价转换成解方程:
	$$
		x = a_1 + m_1t_1 = a_2 + m_2t_2 
	$$
	其中$t_1, t_2$是未知数,我们只需要按照第一题一样的方式解方程,然后将$t_1$的解代入$x = a_1 + m_1t_1$,就能得到$x$的解,并且可以发现解的周期是$[m_1,m_2]$。从而上面的两个方程等价于一个下面形式的方程:
	$$
		x \equiv a_3 ( [m_1,m_2] )
	$$
	其中$a_3$是解得的$x$的一个特解。
	我们这样一直做下去,就要么在一次合并中得出无解,要么就可以将方程组合成成一个方程。
	
	\section{seq}
	很裸的一个线性递推,我们可以构造一个$4 * 4$的矩阵用矩阵快速幂来加速转移,输出后$18$位只需要模$10^{18}$就行,因为模数很大,我们还需要为乘法操作写一个快速乘。
	
	\section{phica}
	由欧拉定理:
	$$
		2^{f(n)} \equiv 2^{f(n) \; mod \; \varphi(107) } \; (mod \; 107)
	$$
	所以我们只需要求出$f(n) \; mod \; 106$即可,显然$f(n)$是Catalan数列,所以:
	$$
		f(n) = \binom{2n}{n} - \binom{2n}{n-1}
	$$
	所以算出$100$以内的组合数就可以搞定了。
\end{document}