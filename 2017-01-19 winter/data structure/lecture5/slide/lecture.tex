\documentclass{beamer}
\usepackage{xeCJK}
\usepackage{listings}
\usepackage{graphicx}
\usepackage{indentfirst}

\title{数据结构第四讲}
\subtitle{STL常用工具}
\author{丁尧尧}
\institute{上海交通大学}
\date{\today}
\usetheme{PaloAlto}
\setlength{\parindent}{1em}

\begin{document}
	\maketitle
	\begin{frame}{目录}
		\tableofcontents
	\end{frame}
	\begin{frame}
		请用好:
	
			
				
		\texttt{http://www.cplusplus.com/reference}
		
		
		
		这个网站
	\end{frame}
	\section{Container}
		\begin{frame}{常用容器}
			常用容器有以下几种,要使用的话需要包含同名的头文件。
			\begin{description}
				\item[vector] 可变长的数组
				\item[map] 本质是棵红黑树,可以充当范围很大的数组
				\item[set] 集合
				\item[list] 双向链表
				\item[deque] 双端队列
			\end{description}
		\end{frame} 
	\section{Algorithm}
		\begin{frame}{常用算法}
			有以下常用算法:
			\begin{description}
				\item[unique] 常用于离散化
				\item[rotate] 转起来!
				\item[sort] 排序
				\item[lower\_bound] 第一个大于等于
				\item[upper\_bound] 第一个大于
				\item[next\_permutation,perv\_permutation] 下一个(上一个)排列
			\end{description}
		\end{frame}
	\section{Utility}
		\begin{frame}{小工具}
				有些有用的小东西
				\begin{description}
					\item[stringstream] 转换
					\item[pair] 方便返回一对东西
					\item[bitset] 位运算
				\end{description}
			\end{frame}
		\end{frame} 
\end{document} 

