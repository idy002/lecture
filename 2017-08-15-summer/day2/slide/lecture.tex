\documentclass[9pt]{beamer}
\usepackage{xeCJK}

\usetheme{Berkeley}
\usefonttheme[onlymath]{serif}
\usepackage[utf8]{inputenc}
\usepackage[T1]{fontenc}
\usepackage{amsmath}
\usepackage{amsfonts}
\usepackage{amssymb}
\usepackage{multicol}


\title{数学第二讲}
\subtitle{离散对数、元根、反演}
\author{丁尧尧}
\institute{上海交通大学}
\date{\today}
\usetheme{PaloAlto}

\begin{document}
	\maketitle
	\begin{frame}{目录}
		\tableofcontents
	\end{frame}
	
	\section{离散对数}
		\begin{frame}
			对于以下问题:
			\begin{definition}[离散对数]
				给定$a$,$b$,$m$,其中$a$与$m$互素,求最小的非负(正)整数$x$,使得:
				$$
				a^x \equiv b \quad (mod \; m)
				$$
				我们称$x$为在模$m$意义下,以$a$为底的$b$的离散对数,记作$ind_ab$。
			\end{definition}
			给出$a,b,m$,(我们假设$a$与$m$互素)我们如何求$x$呢?
		\end{frame}
		\begin{frame}
			有一个叫做大步小步的算法(Baby Step Giant Step)。
			我们假设$x = ic + j$是一个答案(其中$c$是我们自己选定的一个在$2,m-1$之间的数)。
			我们先计算出:
			$$
			a^1, a^2, a^3, \cdots, a^{b-1}
			$$			
			如果发现其中某个值是$b$,那么我们就找到答案了。否则,我们将这些数放在一个数据结构中(平衡二叉树,哈希表都可以),要求可以通过$a^i$的值快速得到$i$。那么我们再依次算出:
			$$
			ba^{-c}, ba^{-2c}, \cdots, ba^{-kc}
			$$
			每算完一个$ba^{-ic}$,我们就看上面的数据结构中是否有一个值$a^j$等于它,如果有,那么它们满足:
			$$
			a^j \equiv ba^{-ic} \quad (mod \; m)
			$$
			即:
			$$
			a^{ic + j} \equiv b \quad (mod \; m)
			$$
		\end{frame}
		\begin{frame}
			分析复杂度,如果我们上面用哈希表存,那么可以$O(1)$判断某个值是否存在。
			
			那么我们总共需要计算的数的个数是$O(b + \frac{m}{b})$,我们取$b = \sqrt{m}$,
			
			可以得到$O(\sqrt{m})$的复杂度.
			
			上面的方法,$a$不限于整数,还可以是矩阵,但都有同一个要求,即$a$存在乘法逆元.
		\end{frame}
	\section{元根}
		\begin{frame}
			我们先介绍一些概念.
			\begin{definition}[剩余系]
				对于给定模数$m(m > 0)$,如果有一组数$\{ a_i\}$:
				$$
					a_1, a_2, a_3,\dots, a_{m}
				$$
				对于任何数$a$,存在唯一数$a_i$满足:
				$$
					a \equiv a_i \quad (mod \; m)
				$$
				那么我们将$\{a_i\}$称作模$m$的一组完全剩余系.
			\end{definition}
			
		\end{frame}
		\begin{frame}
			类似的有:
			\begin{definition}[既约剩余系]
				对于给定模数$m(m > 0)$,如果有一组数$\{ a_i\}$:
				$$
					a_1, a_2, a_3,\dots, a_{k}
				$$
				满足:
				$$
					gcd(a_i,m) = 1
				$$
				且对于任何和$m$互质的数$a$,有唯一的$a_i$满足:
				$$
					a \equiv a_i \quad (mod \; m)
				$$
				那么我们将$\{a_i\}$称作模$m$的一组既约剩余系.
			\end{definition}
			模$m$的既约剩余系的数的个数记作$\varphi(m)$.
		\end{frame}
		\begin{frame}
			\begin{definition}[阶]
				给定一个与$m(1 \leq m)$互素的$a$,则最小的一个满足:
				$$
					a^r \equiv 1 \quad (mod m)
				$$
				的正整数$r$叫做$a$模$m$的阶,一般记作$r = \delta_m(a)$
			\end{definition}
			\begin{definition}[元根] 
				对于模数$m$,如果存在一个数$g$,满足:
				$$
					\delta_m(g) = \varphi(m)
				$$
				我们则称$g$为模$m$的一个元根
			\end{definition}
		\end{frame}
		\begin{frame}
			我们知道,如果$a$与$m$互素,那么:
			$$
				a^1, a^2, a^3,\dots,a^i,\dots
			$$
			都与$m$互素,即它们都是缩系的元素.
			
			元根的意义在于,将缩系中的每一个元素,都与一个$g^i$这种形式对应起来.
			
			假如我们找到了一个模$m$的元根$g$,想求其缩系中的一个元素$a$对应的指数是什么,我们就可以用离散对数找到满足:
			$$
				g^i \equiv a \quad (mod m)
			$$
			的$i$.
		\end{frame}
		\begin{frame}
			并不是所有数都有元根.
			\begin{theorem}
				只有形如:
				$$
				1,2,4,p^{\alpha},2p^{\alpha}
				$$
				的数存在元根(其中$\alpha \geq 1$且$p$是奇素数).
			\end{theorem}
			
			那么我们怎样找元根呢?
			
			\begin{theorem}
				如果存在正数$a,b,m$,且$gcd(a,m) = 1$, 满足
				$$
				a^b \equiv 1 \quad (mod m)
				$$
				那么有:
				$$
				\delta_m(a) \mid b
				$$
			\end{theorem}
			
		\end{frame}
		\begin{frame}
			通过上面这个定理可以证明:
			\begin{theorem}
				对于给定的与$m \geq 2$互素的一个数$g$, $g$是$m$的一个元根当且仅当对于$\varphi(m)$的所有素因子$p_i$,有:
				$$
				g^{\frac{\varphi(m)}{p_i}} \not \equiv 1 \quad (mod \; m)
				$$
			\end{theorem}
			因为在$10^9$范围内的所有素数的最小的元根都很小(最大的不过一百多),所以我们可以暴力从小到大check.
		\end{frame}
		
		\begin{frame}
			我们来道例题看看:
			\begin{block}{例题1}
				给你$a,b,m$,都是正整数,其中$m$是素数,求:
				$$
					a^x \equiv b \quad (mod \; m)
				$$
				其中:$2 \leq m \leq 2\times10^9$,$1 \leq a, b\ < m$
			\end{block}
			很容易离散对数就可以秒了对不对.
		\end{frame}
		\begin{frame}
			\begin{block}{例题2}
				给你$a_1,b_1,a_2,b_2,m$,都是正整数,其中$m$是素数,求满足下面条件的$x$:
				$$
					a_i^x \equiv b_i \quad (mod m) \quad(i = 1, 2)
				$$
				其中:$2 \leq m \leq 2\times10^9$且$1 \leq a_i, b_i < m$.
			\end{block}

		\end{frame}
		\begin{frame}
				先找到$m$的一个元根,然后找到$a_i$和$b_i$的离散对数:
				$$
				g^{c_i} \equiv a_i \quad (mod m)
				$$
				$$
				g^{d_i} \equiv b_i \quad( mod m)
				$$
				然后就把问题化简成了:
				$$
				g^{xc_i} \equiv g^{d_i} \quad( mod m)
				$$
				因为$g$是模$m$的元根,所以上面的方程等价于:
				$$
					xc_i \equiv d_i \quad(mod \quad m - 1)
				$$
				从而把问题转化为解一元一次同余方程组的问题.
			\end{frame}
						
			\begin{frame}
				\begin{block}{例题3}
				给你三个正整数$a,b,m$,其中$m$是质数,求$x$满足:
				$$
						x^a \equiv b \quad (mod m)
				$$
				其中:$1 \leq x, b< m$
				\end{block}
				同样先求离散对数,然后解一次方程,得到$ind_g(x)$,最后快速幂一下就行了.
			\end{frame}
			\begin{frame}
				元根,离散对数,主要的作用是把一些和指数有关的问题转化成一般的一次同余方程,类似于正实数上的开$log$运算.
				
				只是在模意义下,我们的底需要精细地选取.
			\end{frame}
	
		\section{反演}
			\subsection{
			\begin{frame}
				
			\end{frame}
				
\end{document}

