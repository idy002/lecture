\documentclass[10pt,a4paper,twoside]{article}
\usepackage[utf8]{inputenc}
\usepackage{amsmath}
\usepackage{amsfonts}
\usepackage{amssymb}
\usepackage{graphicx}
\usepackage{xeCJK}
\author{idy002}
\title{Solution}

\begin{document}
	\maketitle
	
	\newpage
	\section{modlog}
		\subsection{10\%}
			暴力,不用快速幂。
		\subsection{30\%}
			暴力,用快速幂。
		\subsection{另外30\%}
			解
			$$
				x^a \equiv b \quad (mod \; m)
			$$
			如果$b$为$0$,那么只有$x = 0$一个解。所以以下讨论都建立在$b \not= 1$的基础上。
			首先找到$m$的一个元根$g$,然后求$b$的离散对数$ind_gb$,方程就变成了:
			$$
				aind_gx \equiv ind_gb \quad (mod \; m - 1)
			$$
			这个方程的每个解对应于原方程的一个解。
			
			具体来说,如果$(a,m-1) \not\mid ind_gb$,那么原方程无解,否则有$(a,m-1)$个解。
			 
		\subsection{100\%}
			假如我们解出了上面那个方程:
			$$
				ind_gx \equiv r \quad(mod \; s)
			$$
			其中:
			$$
				s = \frac{m-1}{(m-1,a)}
			$$
			我们的$(m-1,a)$个解就是:
			$$
				g^{r}, g^{r+s}, g^{r+2s},\dots,g^{r+m-1-s}
			$$
			如果公比是$1$,即$g^s = 1 \quad (mod \; m)$,那么和就是$(m-1,a)g^r$。
			
			如果公比不是$1$,那么由等比数列求和更是我们会发现答案为$0 \quad (mod \; m)$。
	\newpage
	\section{sumit}
		\subsection{30\%}
			枚举暴力。复杂度$O(n^2)$。
		\subsection{另外30\%}
			我们推一推反演公式,发现结果如下:
			$$
				ans = \sum_{d = 1}^{min(n,m)} \mu(d) d F(\lfloor \frac{n}{d} \rfloor, \lfloor \frac{m}{d} \rfloor)
			$$
			其中:
			$$
				F(a,b) = b\sum_{i = 1}^{b}i + a\sum_{i = 1}^{a}i
			$$
			显然$F(a,b)$可以直接算,于是我们就暴力枚举$d$即可,复杂度$O(n)$。
			
		\subsection{100\%}
			上面对$\mu(d)d$求一个前缀和,我们就可以一块一块地跳。$O(\sqrt{n})$
		
	\newpage
	\section{secret}
		\subsection{20\%}
			dfs暴力枚举每个数。
		\subsection{60\%}
			我们因为有:
			  $$
				   a_i = \sum_{i \mid j} b_j
			  $$
			所以:
			$$
				b_i = a_i - \sum{i \mid j \text{且} j \neq i} b_j
			$$
			然后从后往前算,就可以了。因为中间有个调和级数,所以复杂度$O(nlogn)$。
		\subsection{100\%}
			由莫比乌斯反演的另一种形式,我们有:
			$$
				b_i = \sum_{i \mid j}a_j\mu(\frac{j}{i})
			$$	
			讲这里的$i$代成$1$即可。复杂度$O(n)$。
\end{document}