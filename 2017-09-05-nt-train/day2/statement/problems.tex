\documentclass[11pt,a4paper,oneside]{article}
\usepackage[english]{babel}
\usepackage{olymp}
\usepackage[dvips]{graphicx}
\usepackage{color}
\usepackage{colortbl}
%\usepackage{expdlist}
%\usepackage{mfpic}
%\usepackage{comment}
\usepackage{multirow}
\usepackage{amsmath}
\usepackage{xeCJK}

%\setCJKmainfont[BoldFont={Hei}]
%{SimSun}
%\setCJKmonofont{FangSong}

\renewcommand{\contestname}{
No.7 High School Summer Training \\
idy002, \today
}    

\begin{document}


\begin{problem}{facsum}{facsum.in}{facsum.out}{2 second}{256 MB}	
	
	Mr.Hu最近偶得一函数:
	
	$$
		f(n) = \big (\sum_{d \mid n} \varphi(d) \big )^m \big (\sum_{d \mid n} \sigma_0(d)\mu(\frac{n}{d})\frac{n}{d} \big)
	$$
	
	
	其中$\sigma_0(n)$表示$n$的正约数个数,比如$\sigma_0(12) = 6$,因为$12$有$1,2,3,4,6,12$共$6$个正约数。
	
	其中$\varphi(n)$是欧拉函数,$\mu(n)$是莫比乌斯函数。
	
	又有:
	
	$$
		F(n) = \sum_{i = 1}^{n} f(i)
	$$
	
	Mr.Hu希望你计算$F(n) \quad mod \; 10^9 + 7$的值。
	
	\InputFile
	
	第一行包含两个整数:$n \; m$。
	
	\OutputFile
	
	输出一行包含一个数,表示答案。

	\Example
	
	\begin{example}
		\exmp{
			3 1
		}{
			1000000005
		}%
	\end{example}
	
	样例解释:$f(1) = 1 \quad f(2) = 0 \quad f(3) = -3$,故$F(3) = f(1) + f(2) + f(3) = -2$,在模意义下,这个数为:$1000000005$。
	
	\Note
	
	\begin{itemize}
		\item 对于$20\%$的数据,$1 \leq n \leq 5000$。
		\item 对于$50\%$的数据,$1 \leq n \leq 10^5$。
		\item 对于$100\%$的数据,$1 \leq n \leq 10^7$,$1 \leq m \leq 10$。
	\end{itemize}

\end{problem}

\begin{problem}{group}{group.in}{group.out}{1 second}{256 MB}	
	
	Mr.Hu最近在研究等比数列,即形如:
	$$
		a, a^1, a^2, a^3, \dots, a^n, \dots
	$$
	现在,Mr.Hu想知道,对于给定的非负整数$a$,上面这个无穷数列在摸$mod$意义下有多少项是本质不同的。(保证$gcd(a,mod) = 1$)。
	
	\InputFile
	
	第$1$行一个整数:$T$,表示数据组数。
	
	接下来$T$行,每行两个整数:$a \; mod$。
	
	\OutputFile
	
	对于每组数据,输出一行,包含一个整数,表示模意义下本质不同的数有多少个。
	
	\Example
	
	\begin{example}
		\exmp{
			2
			1 3
			2 5
		}{
			1
			4
		}%
	\end{example}

	对于第一组数据,数列是:$1, 1, 1, \dots, 1, \dots$
	
	对于第二组数据,数列(取模以后)是:$2, 4, 3, 1, 2, 4, 3, 1, \dots$,总共有$4$个本质不同的数。
	
	\Note

	\begin{itemize}
		\item 对于$30\%$的数据,$0 \leq a \leq 10^3$,$1 \leq mod \leq 10^3$;
		\item 对于$100\%$的数据,$0 \leq a \leq 2 \times 10^9$,$1 \leq mod \leq 2 \times 10^9$,且保证$gcd(a,mod) = 1$,$1 \leq T \leq 100$。
	\end{itemize}

\end{problem}

\begin{problem}{ccount}{ccount.in}{ccount.out}{1 second}{256 MB}	
	
	Mr.Hu最近在学习组合数,他觉得这些数非常美丽。
	
	于是,他写下了这样一个数列:
	
	$$
		\binom{n}{l}, \binom{n}{l+1}, \binom{n}{l+2}, \dots, \binom{n}{r-1}, \binom{n}{r}
	$$
	
	Mr.Hu想知道,这些数里面,有多少个数是$5$的倍数。
	
	\InputFile
	
	第$1$行一个整数:$T$,表示数据组数。
	
	接下来$T$行,每行三个整数:$l \; r \; n$。
	
	\OutputFile
	
	对于每组数据,输出一行,包含一个整数,表示答案。
	
	\Example
	
	\begin{example}
		\exmp{
			2
			1 3 4
			1 4 5
		}{
			0
			4
		}%
	\end{example}
	
	对于第一组数据,数列是:$4 \; 6 \; 4$,没有$5$的倍数,故答案为$0$。
	
	对于第二组数据,数列是:$5 \; 10 \; 10 \; 5$,有$4$个数是$5$的倍数,故答案为$4$。
	
	\Note
	
	\begin{itemize}
		\item 对于$20\%$的数据,$1 \leq n \leq 5000$。
		\item 对于$40\%$的数据,$1 \leq n \leq 10^9$,$1 \leq r - l + 1 \leq 5000$。
		\item 对于$100\%$的数据,$1 \leq n \leq 10^{18}$,$0 \leq l \leq r \leq n$,$1 \leq T \leq 100$。
	\end{itemize}

\end{problem}

\end{document}
