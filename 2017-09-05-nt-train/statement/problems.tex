\documentclass[11pt,a4paper,oneside]{article}
\usepackage[english]{babel}
\usepackage{olymp}
\usepackage[dvips]{graphicx}
\usepackage{color}
\usepackage{colortbl}
%\usepackage{expdlist}
%\usepackage{mfpic}
%\usepackage{comment}
\usepackage{multirow}
\usepackage{amsmath}
\usepackage{xeCJK}

%\setCJKmainfont[BoldFont={Hei}]
%{SimSun}
%\setCJKmonofont{FangSong}

\renewcommand{\contestname}{
No.7 High School Summer Training \\
idy002, \today
}    

\begin{document}

\begin{problem}{pay}{pay.in}{pay.out}{1 second}{256 MB}

	Mr.Hu 开了个饭店,来了两位客人:Alice和Bob,他们吃完饭要结账时,发现他们需要支付c元钱,但是Alice只有面值为a的钱,Bob只有面值为b的钱(他们每个人的钱的和都大于c,即可以认为他们有无数张对应面值的钱)。现在,Mr.Hu想知道,他们可能刚好支付完饭钱吗?如果可能,那么有多少种方式?你还需要计算出他们所有可能的支付方式的支付的钱的张数的和。

    \InputFile

    第$1$行包含$1$个整数:$T \; opt$,其中$T$表示数据组数,$opt$为数据类型。

	接下来$T$行,每行$3$个整数:$a \; b\; c$。

    \OutputFile
    
    对于每组数据:
    \begin{itemize}
    	\item 如果$opt = 1$,输出一行,包含一个整数:$A$,其中$A$表示刚好支付的方案数。
    	\item 如果$opt = 2$,输出一行,包含两个整数:$A \; B$,其中$A$表示刚好支付的方案数,$B$表示所有可能支付方式的张数和。
    \end{itemize}
    
    \Example

    \begin{example}
        \exmp{
        	2 2
        	3 4 21
        	2 4 12
        }{
			2 13 
			4 18
        }%
    \end{example}
	
	样例解释:
	
	对于$3 \; 4 \; 21$,一共有两种可能的支付方式,分别是:$(3,3), (7,0)$\footnote{其中$(x,y)$表示Alice支付$x$张面值为a的钱,Bob支付$y$张面值为b的钱},所以$A$为$2$,$B$为$3 + 3 + 7 + 0 = 13$。
	
	对于$2 \; 4 \; 12$,一共有四种可能的支付方式,分别是:$(6,0),(4,1),(2,2),(0,3)$,所以$A$为$4$,$B$为$6 + 0 + 4 + 1 + 2 + 2 + 0 + 3 = 18$。
	
    \Note
    
    \begin{itemize}
		\item 对于$20\%$的数据,$1 \leq a, b, c \leq 10000$,$1 \leq T \leq 1000$;
		\item 对于另外$40\%$的数据,$1 \leq a, b, c \leq 10^9$,其中$opt = 1$;
		\item 对于另外$40\%$的数据,$1 \leq a, b, c \leq 10^9$,其中$opt = 2$;
		\item 对于$100\%$的数据,$1 \leq T \leq 10^5$,$1 \leq opt \leq 2$。
    \end{itemize}

\end{problem}

\begin{problem}{sumcomb}{sumcomb.in}{sumcomb.out}{1 second}{256 MB}
	
	Mr.Hu被传送到了一个无限大的表格上,现在这个表格的第$i$行第$j$列的值是$a_{i,j}$($0 \leq i, j$):
	$$
		a_{i,j} = 
		\begin{cases}
			1 & j = 0 \text{或} i = j \\
			a_{i-1,j} + a_{i-1,j-1} & 0 < j < i \\
			0 & j > i
		\end{cases}
	$$
	
	现在,Mr.Hu站在$(n,m)$这个位置,他想知道,他向上或向左上方45度望去,看到的数的和是多少。
	
	从$(n,m)$向上望去,他会看到$(n,m),(n-1,m),(n-2,m),\cdots,(0,m)$这些位置。
	
	从$(n,m)$向左上方45度望去,他会看到$(n,m),(n-1,m-1),\cdots$,直到某一维的下标变为0.
	
	这个数可能很大,你只需将答案对$10^9 + 7$取模即可。
	
	\InputFile
	
	第$1$行一个整数:$T$,表示数据组数。
	
	接下来$T$行,每行格式为:$dir \; n \; m$,其中$dir$为$1$表示向上看,$2$表示向左上方看,$(n,m)$为Mr.Hu现在的位置。
	
	\OutputFile
	
	对于每组数据,输出一行表示答案。
	
	\Example
	
	\begin{example}
		\exmp{
			2
			1 3 2
			2 3 2
		}{
			4
			6
		}%
	\end{example}
	
	表格左上角长成这样(行列都是0 base的):
	$$
		1 \; 0 \; 0 \; 0
	$$
	$$
		1 \; 1 \; 0 \; 0
	$$
	$$
		1 \; 2 \; 1 \; 0
	$$
	$$
		1 \; 3 \; 3 \; 1
	$$
	这样从$(3,2)$向上看,会看到:$3 \; 1 \; 0 \; 0$,和为$4$。
	
	向左上角看,会看到:$3 \; 2 \; 1$,和为$6$。
	
	\Note

	\begin{itemize}
		\item 对于$30\%$的数据,$1 \leq n, m \leq 5000$;
		\item 对于$100\%$的数据,$1 \leq n, m \leq 10^6$,$1 \leq T \leq 50000$。
	\end{itemize}

\end{problem}

\begin{problem}{kor}{kor.in}{kor.out}{1 second}{256 MB}	
	
	Mr.Hu觉得在学习过程中,需要举一反三,做一题要理解透,然后遇到相似的问题时能类似地转化。所以想了一道和以前类似的题目,相信聪明如你,肯定能轻而易举地解决。
	
	Mr.Hu会给你$n$个非负整数,然后从中选$k$个出来,然后把这$k$个数按位或起来,Mr.Hu想知道有多少种选法,使得或起来的结果为$r$。
	
	\InputFile
	
	第$1$行一个整数$T$,表示测试组数。
	
	接下来$T$组数据,对于每组数据:
	
	第$1$行两个整数$n \; k \; r$。
	
	接下来$1$行包含$n$个非负整数:$a_1 \; a_2 \; \dots \; a_n$。
	
	\OutputFile
	
	对于每组数据,输出一行,包含一个整数,即方案数,因为结果可能很大,只需要对$10^9 + 7$取模即可。
	
	\Example
	
	\begin{example}
		\exmp{
			2
			4 2 3
			1 2 3 4 
			4 1 1
			1 2 3 4
		}{
			3
			1
		}%
	\end{example}
	
	对于第一组数据,一共有$3$种选法:$(1,2),(1,3),(2,3)$。
	
	对于第二组数据,一共有$1$种选法:$(1)$。
	
	\Note
	
	\begin{itemize}
		\item 对于$10\%$的数据,$1 \leq n \leq 10$,$0 \leq a_i < 2^{10}$;
		\item 对于$30\%$的数据,$1 \leq n \leq 100$,$0 \leq a_i < 2^{10}$;
		\item 对于$50\%$的数据,$1 \leq n \leq 10^5$,$0 \leq a_i < 2^{15}$;
		\item 对于$100\%$的数据,$1 \leq n \leq 10^5$,$0 \leq a_i < 2^{20}$,$1 \leq k \leq n$,$1 \leq T \leq 5$。
	\end{itemize}
	
\end{problem}


\begin{problem}{facsum}{facsum.in}{facsum.out}{2 second}{256 MB}	
	
	Mr.Hu最近偶得一函数:
	
	$$
		f(n) = \big (\sum_{d \mid n} \varphi(d) \big )^m \big (\sum_{d \mid n} \sigma_0(d)\mu(\frac{n}{d})\frac{n}{d} \big)
	$$
	
	
	其中$\sigma_0(n)$表示$n$的正约数个数,比如$\sigma_0(12) = 6$,因为$12$有$1,2,3,4,6,12$共$6$个正约数。
	
	其中$\varphi(n)$是欧拉函数,$\mu(n)$是莫比乌斯函数。
	
	又有:
	
	$$
		F(n) = \sum_{i = 1}^{n} f(i)
	$$
	
	Mr.Hu希望你计算$F(n) \quad mod \; 10^9 + 7$的值。
	
	\InputFile
	
	第一行包含两个整数:$n \; m$。
	
	\OutputFile
	
	输出一行包含一个数,表示答案。

	\Example
	
	\begin{example}
		\exmp{
			3 1
		}{
			1000000005
		}%
	\end{example}
	
	样例解释:$f(1) = 1 \quad f(2) = 0 \quad f(3) = -3$,故$F(3) = f(1) + f(2) + f(3) = -2$,在模意义下,这个数为:$1000000005$。
	
	\Note
	
	\begin{itemize}
		\item 对于$20\%$的数据,$1 \leq n \leq 5000$。
		\item 对于$50\%$的数据,$1 \leq n \leq 10^5$。
		\item 对于$100\%$的数据,$1 \leq n \leq 10^7$,$1 \leq m \leq 10$。
	\end{itemize}

\end{problem}

\begin{problem}{group}{group.in}{group.out}{1 second}{256 MB}	
	
	Mr.Hu最近在研究等比数列,即形如:
	$$
		a, a^1, a^2, a^3, \dots, a^n, \dots
	$$
	现在,Mr.Hu想知道,对于给定的非负整数$a$,上面这个无穷数列在摸$mod$意义下有多少项是本质不同的。(保证$gcd(a,mod) = 1$)。
	
	\InputFile
	
	第$1$行一个整数:$T$,表示数据组数。
	
	接下来$T$行,每行两个整数:$a \; mod$。
	
	\OutputFile
	
	对于每组数据,输出一行,包含一个整数,表示模意义下本质不同的数有多少个。
	
	\Example
	
	\begin{example}
		\exmp{
			2
			1 3
			2 5
		}{
			1
			4
		}%
	\end{example}

	对于第一组数据,数列是:$1, 1, 1, \dots, 1, \dots$
	
	对于第二组数据,数列(取模以后)是:$2, 4, 3, 1, 2, 4, 3, 1, \dots$,总共有$4$个本质不同的数。
	
	\Note

	\begin{itemize}
		\item 对于$30\%$的数据,$0 \leq a \leq 10^3$,$1 \leq mod \leq 10^3$;
		\item 对于$100\%$的数据,$0 \leq a \leq 2 \times 10^9$,$1 \leq mod \leq 2 \times 10^9$,且保证$gcd(a,mod) = 1$,$1 \leq T \leq 1000$。
	\end{itemize}

\end{problem}

\begin{problem}{ccount}{ccount.in}{ccount.out}{1 second}{256 MB}	
	
	Mr.Hu最近在学习组合数,他觉得这些数非常美丽。
	
	于是,他写下了这样一个数列:
	
	$$
		\binom{l}{n}, \binom{l+1}{n}, \binom{l+2}{n}, \dots, \binom{r-1}{n}, \binom{r}{n}
	$$
	
	Mr.Hu想知道,这些数里面,有多少个数是$5$的倍数。
	
	\InputFile
	
	第$1$行一个整数:$T$,表示数据组数。
	
	接下来$T$行,每行三个整数:$l \; r \; n$。
	
	\OutputFile
	
	对于每组数据,输出一行,包含一个整数,表示答案。
	
	\Example
	
	\begin{example}
		\exmp{
			2
			1 3 4
			1 4 5
		}{
			0
			4
		}%
	\end{example}
	
	对于第一组数据,数列是:$4 \; 6 \; 4$,没有$5$的倍数,故答案为$0$。
	
	对于第二组数据,数列是:$5 \; 10 \; 10 \; 5$,有$4$个数是$5$的倍数,故答案为$4$。
	
	\Note
	
	\begin{itemize}
		\item 对于$20\%$的数据,$1 \leq n \leq 5000$。
		\item 对于$40\%$的数据,$1 \leq n \leq 10^9$,$1 \leq r - l + 1 \leq 1000$。
		\item 对于$100\%$的数据,$1 \leq n \leq 10^{18}$,$0 \leq l \leq r \leq n$,$1 \leq T \leq 100$。
	\end{itemize}

\end{problem}

\end{document}
