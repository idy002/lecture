\documentclass[11pt,a4paper,oneside]{article}
\usepackage[english]{babel}
\usepackage{olymp}
\usepackage[dvips]{graphicx}
\usepackage{color}
\usepackage{colortbl}
%\usepackage{expdlist}
%\usepackage{mfpic}
%\usepackage{comment}
\usepackage{multirow}
\usepackage{amsmath}
\usepackage{amsfonts}
\usepackage{amssymb}
\usepackage{xeCJK}

%\setCJKmainfont[BoldFont={Hei}]
%{SimSun}
%\setCJKmonofont{FangSong}

\renewcommand{\contestname}{
2018清北冬令营模拟测试
}    

\begin{document}

\begin{problem}{chess}{chess.in}{chess.out}{2 second}{512}
	
	一个$n$行$m$列的棋盘,我们准备在上面放一些马,并且让马与马之间不能相互攻击(马的攻击方式是中国象棋中马的攻击方式,并且不考虑“蹩马脚”的情况,即任何两匹马,如果他们在某一维位置相差1,在另一维相差2,则他们可以相互攻击),但是有些位置坏掉了,所以不能放马,你需要计算最多能放多少匹马,并且给出一种马的放法。
	
	\InputFile
	
	第一行四个整数$n, m,q,type$,表示棋盘大小及坏掉的位置的个数,$type$表示数据类型;

	接下来$q$行,每行两个整数$x,y$表示$x$行$y$列的位置坏掉了。
	
	\OutputFile
	
	第一行包含一个整数$ans$,表示最多还可以放多少匹马;
	
	如果$type = 0$,则不再输出信息。
	
	如果$type = 1$,则还需要输出$ans$行,每行两个整数$x,y$ ,表示一匹马的位置。
	
	\Example
	
	样例数据1
	
	\begin{example}
		\exmp{
			2 2 2 1
			1 1
			2 2
		}{
			2
			1 2
			2 1
		}%
	\end{example}
	
	样例数据2见sample/chess2.in和sample/chess2.out。
	
	\Note
	
	注:本题有Special Judge。
	
	\begin{itemize}
		\item 对于$30\%$的数据,$1 \leq n, m \leq 10$, $q = 0$, $type = 1$;
		\item 对于另外$30\%$的数据,$1 \leq n, m \leq 30$, $type = 0$;
		\item 对于$100\%$的数据,$1 \leq n, m \leq 30$, $0 \leq q \leq nm$, $0 \leq type \leq 1$, $1 \leq x \leq n$, $1 \leq y \leq m$。
	\end{itemize}
\end{problem}

\begin{problem}{tower}{tower.in}{tower.out}{2 second}{512}
	
	现在有一条 $[1, l]$ 的数轴,要在上面造 $n$ 座塔,每座塔的坐标要两两不
	同,且为整点。塔有编号,且每座塔都有高度,对于编号为 $i$ 座塔,其高
	度为 $i$。对于一座塔,需要满足它与前面以及后面的塔的距离大于等于自身
	高度(不存在则没有限制)。问有多少建造方案。答案对 $m$ 取模。塔不要
	求按编号为顺序建造。
	
	\InputFile
	
	一行三个整数 $n, l, m$。
	
	\OutputFile
	
	输出一个整数,代表答案对 $m$ 取模的值。
	
	\Example
	
	样例数据1
	
	\begin{example}
		\exmp{
			3 9 17
		}{
			15
	}%	
	\end{example}
	
	样例数据2见sample/tower2.in和sample/tower2.out。

	\Note
	\begin{itemize}
		\item 对于 $10\%$ 的数据,满足:$n \leq  10, l \leq 25$
		\item 对于 $30\%$ 的数据,满足:$n \leq 20$
		\item 对于 $50\%$ 的数据,满足:$n \leq 50$
		\item 对于 $70\%$ 的数据,满足:$l \leq 10^5$
		\item 对于 $100\%$ 的数据,满足:$n \leq 100, 1 \leq l \leq 10^9 , 1 \leq m \leq 10^9$
	\end{itemize}

\end{problem}

\begin{problem}{stream}{stream.in}{stream.out}{2 second}{512}
	
	点点太无聊了,所以想学习学习最大流来打发无聊的时光,所谓最大流,
	是指图上的每一条边都有一定的流量限制,每个顶点处要满足流量的收支平
	衡,最终目标是最大化进入汇点的流量。接下来我们就来考虑一个动态改变源
	汇点的最大流问题。
	给出一个无向简单(无重边自环)连通图,由 $N$ 个点 $M$ 条边构成,保证
	每个点最多属于一个简单环。共有 $Q$ 次操作,每次或询问给定源汇点的最大
	流、或修改一条边的流量。现在请你按顺序回答每次询问的答案。
	
	\InputFile
	
	第一行两个整数 $N,M$;
	
	接下来 $M$ 行,每行三个整数 $u,v,f$,表示 $u,v$ 间有一条流量限制为 $f$
	的边;
	
	接下来一行一个整数$Q$;
	
	再之后 $Q$ 行每行三个整数,若形如“0 S T”则表示求源点为 S、汇点为
	T 时的最大流;若形如“1 x f”,则表示把编号为 x(按读入顺序从 1 开
	始)的边流量限制改为 f。
	
	\OutputFile
	
	对于每组询问,输出一行一个整数,表示最大流。
	
	\Example
	
	\begin{example}
		\exmp{
			4 3
			1 2 2
			2 3 1
			2 4 3
			3
			0 1 3
			1 1 5
			0 1 4
		}{
			1
			3
	   }%
	\end{example}

	\Note
	\begin{itemize}
		\item 有$10\%$的数据,$N,M,Q \leq 1000,M=N-1$;
		\item 有$20\%$的数据,$N,M,Q \leq 100000,M=N-1$;
		\item 另$10\%$的数据,$N,M,Q \leq 100$;
		\item 另$30\%$的数据(含前一类),$N,M,Q \leq 1000$;
		\item 对于$100\%$的数据,$N \leq 100000,M,Q \leq 200000$,边的流量限制始终为不超过$10^9$的正整数,保证询问中S和T不等。
	\end{itemize}
\end{problem}

\end{document}