\documentclass[11pt,a4paper,oneside]{article}
\usepackage[english]{babel}
\usepackage{olymp}
\usepackage[dvips]{graphicx}
\usepackage{color}
\usepackage{colortbl}
%\usepackage{expdlist}
%\usepackage{mfpic}
%\usepackage{comment}
\usepackage{multirow}
\usepackage{amsmath}
\usepackage{amsfonts}
\usepackage{amssymb}
\usepackage{xeCJK}

%\setCJKmainfont[BoldFont={Hei}]
%{SimSun}
%\setCJKmonofont{FangSong}

\renewcommand{\contestname}{
2018清北冬令营模拟测试
}    

\begin{document}

\begin{problem}{energy}{energy.in}{energy.out}{2 second}{512}
	
	有一块神秘的大陆,大陆的地图可以被视为 $n \times m$ 的格点图。其中每一个格
	点里都有一块矿石,每个矿石都有自己的能量值,第 $i$ 行第 $j$ 列的矿石能量被
	记为 $p_{i,j}$ 。
	
	现在有一个能量收集装置,它可以被放置在任何一个格点上,周围矿石的能
	量都会有一部分被吸收。具体来说如果位于坐标 $(i, j)$ 矿石与收集装置的欧式距
	离 $d$ 小于 $r$ 的话,那么这块矿石将会有 $\frac{p_{i,j}}{1+d}$
	的能量被采集。
	
	请问最多能采集多少的矿石能量?
	
	\InputFile
	
	第一行为两个整数 $n, m$,和一个浮点数 $r$。
	接下来 $n$ 行,每行 $m$ 个浮点数,第 $i$ 行第 $j$ 个数表示 $p_{i,j}$ 。

	\OutputFile
	
	一行一个浮点数,表示答案,四舍五入到三位小数部分。
	
	\Example
	
	\begin{example}
		\exmp{
			5 5 1.5
			4 3 2 9 1
			3 4 3 2 8
			9 4 3 2 1
			2 3 0 1 2
			6 3 4 3 1
		}{
			17.956
		}%
	\end{example}

	\Note
	
	\begin{itemize}
		\item 对于 $30\%$ 的数据,$1 \leq n \leq 100$, $1 \leq m \leq 100$, $0 \leq r \leq 50$。
		\item 对于 $100\%$ 的数据,$1 \leq n \leq 500$, $1 \leq m \leq 500$, $0 \leq r \leq 300$。
	\end{itemize}

\end{problem}

\begin{problem}{ernd}{ernd.in}{ernd.out}{2 second}{512}
	对于一棵有根树,定义一个点 $u$ 的 k-子树为 $u$ 的子树中距离 $u$ 不超
	过 $k$ 的部分。注意,假如 $u$ 的子树中不存在距离 $u$ 为 $k$ 的点,则 $u$ 的 k-
	子树是不存在的。
	
	定义两棵子树是相同的,当且仅当不考虑点的标号时,他们的形态是
	相同的(儿子的顺序也需要考虑)。给定一棵 $n$ 个点,点的标号在 $[1, n]$,
	以 $1$ 为根的有根树。
	
	问最大的 $k$,使得存在两个点 $u\neq v$,满足 $u$ 的 k-子
	树与 $v$ 的 k-子树相同。

	\InputFile
	
	第一行输入一个正整数 $n$。
	
	接下来读入 $n$ 个部分,第 $i$ 个部分描述点 $i$ 的儿子,且以顺序给出。
	
	每个部分首先读入一个整数 $x$,代表儿子个数。接下来 $x$ 个整数,代
	表从左到右儿子的标号。
	
	\OutputFile
	
	输出一个整数 $k$,代表最大的合法的 $k$。
	
	\Example
	
	样例数据1
	
	\begin{example}
		\exmp{
			8 
			1
			2
			2
			3 4
			0
			1
			5
			2
			6 7
			0
			1
			8
			0
		}{
			3
		}%
	\end{example}
	
	样例数据2见sample/ernd2.in和sample/ernd2.out。
	
	\Note
	
	\begin{itemize}
		\item 对于 $20\%$ 的数据,满足:$n \leq 100$
		\item 对于 $40\%$ 的数据,满足:$n \leq 2000$
		\item 对于 $60\%$ 的数据,满足:$n \leq 30000$
		\item 对于 $100\%$ 的数据,满足:$n \leq 100000$,保证给出的树是合法的。
	\end{itemize}
	
\end{problem}


\begin{problem}{gene}{gene.in}{gene.out}{5 second}{512}
	
	给定一个长度为 $n$ 的字符串 $s$,有 $q$ 组询问,每个询问给定 $l, r$,询问
	$s[l..r]$ 中有多少本质不同的回文子串。
	
	\InputFile
	
	第一行一个整数 $type$,若 $type = 0$,表示这个数据没有进行加密,若
	$type = 1$,表示这个数据进行了加密。
	
	第二行两个整数 $n, q$。
	
	第三行一个字符串 $s$。
	
	接下来 $q$ 行,每行两个整数 $l' , r'$ 。记 $lastans$ 为上一次询问的答案,
	若这是第一次询问,$lastans = 0$,则这次询问的 $l, r$ 为$l = l'\oplus (type \times
	lastans)$, $r = r'\oplus (type \times lastans)$。
	
	\OutputFile
	
	输出共 $q$ 行,代表每个询问的答案。
	
	\Example
	
	样例数据1
	
	\begin{example}
		\exmp{
			1 
			8 4 
			abbabbba 
			1 7 
			3 2
			6 10
			1 0
		}{
			7
			2
			5
			2
		}%
	\end{example}
	
	样例数据2见sample/gene2.in和sample/gene2.out。
	
	\Note
	
	% Please add the following required packages to your document preamble:
	% \usepackage{multirow}
	\begin{table}[!htbp]
		\begin{tabular}{|c|c|c|}
			\hline
			数据点 & n不超过                     & type                \\ \hline
			1   & 100                      & \multirow{4}{*}{1}  \\ \cline{1-2}
			2   & \multirow{3}{*}{1000}    &                     \\ \cline{1-1}
			3   &                          &                     \\ \cline{1-1}
			4   &                          &                     \\ \hline
			5   & \multirow{3}{*}{30000}   & \multirow{6}{*}{0}  \\ \cline{1-1}
			6   &                          &                     \\ \cline{1-1}
			7   &                          &                     \\ \cline{1-2}
			8   & \multirow{13}{*}{100000} &                     \\ \cline{1-1}
			9   &                          &                     \\ \cline{1-1}
			10  &                          &                     \\ \cline{1-1} \cline{3-3} 
			11  &                          & \multirow{10}{*}{1} \\ \cline{1-1}
			12  &                          &                     \\ \cline{1-1}
			13  &                          &                     \\ \cline{1-1}
			14  &                          &                     \\ \cline{1-1}
			15  &                          &                     \\ \cline{1-1}
			16  &                          &                     \\ \cline{1-1}
			17  &                          &                     \\ \cline{1-1}
			18  &                          &                     \\ \cline{1-1}
			19  &                          &                     \\ \cline{1-1}
			20  &                          &                     \\ \hline
		\end{tabular}
	\end{table}

\end{problem}


\end{document}