\documentclass[11pt,a4paper,oneside]{article}
\usepackage[english]{babel}
\usepackage{olymp}
\usepackage[dvips]{graphicx}
\usepackage{color}
\usepackage{colortbl}
%\usepackage{expdlist}
%\usepackage{mfpic}
%\usepackage{comment}
\usepackage{multirow}
\usepackage{amsmath}
\usepackage{amsfonts}
\usepackage{amssymb}
\usepackage{xeCJK}

%\setCJKmainfont[BoldFont={Hei}]
%{SimSun}
%\setCJKmonofont{FangSong}

\renewcommand{\contestname}{
No.7 High School Number Theory Special Training \\
idy002, \today
}    

\begin{document}

\begin{problem}{eqution}{eqution.in}{eqution.out}{1 second}{256}
	
	Mr. Hu又来让你帮忙解方程了。
	
	方程是这样的:
	$$
		x_1 + x_1 + x_3 + \dots + x_n = m \quad ( x_i \geq 0 \;  \forall 1 \leq i \leq n  )
	$$
	Mr. Hu希望你求出这个$n$元一次方程的整数解有多少个,因为解的个数有可能变得很大,所以Mr. Hu只需要你输出解的个数取模于$mod$。
	
	\InputFile
	
	第$1$行,包含一个整数:$T$,表示询问个数
	
	接下来$T$行,每行包含三个整数:$n \; m \; mod$
	
	\OutputFile
	
	输出$T$行,每行输出解的个数模对应$ mod $
	
	\Example
	
	\begin{example}
		\exmp{
			1
			2 3 13
		}{
			4 
		}%
\end{example}

\Note
	样例中,解分别是:$(3,0) , (2,1)  , (1,2) , (0,3)$	
	\begin{itemize}
		\item 对于$30\%$的数据,$ 1 \leq n, m \leq 6 $,$ mod = 10^8 + 7 $,$ T = 1 $
		\item 对于$70\%$的数据,$ 1 \leq n, m \leq 10^3 $,$ n + m \leq mod \leq 10^8 + 7 $,$mod$是一个素数,$ 1 \leq T \leq 100 $
		\item 对于余下$30\%$的数据,$ 1 \leq n, m \leq 10^3$,$ n + m \leq p, q \leq 10^4 $,$mod = pq$,$p, q$是素数,$ 1 \leq T \leq 10^3 $
	\end{itemize}
\end{problem}

\begin{problem}{power}{power.in}{power.out}{1 second}{256}
	
	Mr. Hu打算考一道比较显然的题目,低头一想,就有了这道题。
	
	Mr. Hu需要你计算:
	
	$$
	3^n \; mod \; 10^9 + 8 
	$$
	
	是不是很简单啊。\verb|^_^|
	
	\InputFile
	
	只有一行,一个数$n$。
	
	\OutputFile
	
	输出结果。
	
	\Example
	
	\begin{example}
		\exmp{
			3
		}{
			27
	}%
\end{example}

\Note
\begin{itemize}
	\item 对于$10\%$的数据,$1 \leq n \leq 10^6$
	\item 对于$30\%$的数据,$1 \leq n \leq 10^{18}$
	\item 对于$70\%$的数据,$1 \leq n \leq 10^{1000}$
	\item 对于$100\%$的数据,$1 \leq n \leq 10^{100000}$
\end{itemize}

\end{problem}

\begin{problem}{comb}{comb.in}{comb.out}{1 second}{256}
	
	一天,Mr. Hu对组合数产生了兴趣,他想要知道满足下面条件的数$i$有多少个:
	$$
		gcd(\binom{n}{i},p) = p \quad ( 0 \leq i \leq n )
	$$
	其中$p$是素数。
	
	\InputFile
	
	第$1$行,$2$个整数:$n \; p$。
	
	\OutputFile
	
	输出所求。
	
	\Example
	
	\begin{example}
		\exmp{
			5  2
		}{
			2
	   }%
\end{example}

\Note
	样例中,一共有$6$个组合数:$1, 5, 10, 10, 5, 1$,其中和$2$公约数为$2$的数有两个$10$,故输出$2$。
	\begin{itemize}
		\item 对于$30\%$的数据,满足$ 1 \leq n, p \leq 10^3 $
		\item 对于$100\%$的数据,满足$ 1 \leq n, p \leq 10^{18} $
	\end{itemize}
\end{problem}

\begin{problem}{derange}{derange.in}{derange.out}{1 second}{256}
	
	Mr. Hu需要给机房的$n$位同学重新安排位置,为了使换位的效果好,需要保证每位同学在排位前和排位后所在的位置都不同。
	
	Mr. Hu想知道有多少种可能的排位方法。因为排位数可能很大,所以你只需要输出它模$10^9 + 7$的结果。
	
	\InputFile
	
	输入文件中只有$1$个数$n$,表示学生人数。
	
	\OutputFile
	
	输出可能的排位数取模后的结果。
	
	\Example
	
	\begin{example}
		\exmp{
			3
		}{
			2
	    }%
\end{example}

\Note
	样例说明:假设一开始三个人的位置为$(1,2,3)$,那么合法的换位结果为:$(2,3,1)$和$(3,1,2)$。
	\begin{itemize}
		\item 对于$30\%$的数据,$ 1 \leq n \leq 9 $
		\item 对于$70\%$的数据,$ 1 \leq n \leq 10^3 $
		\item 对于$100\%$的数据,$ 1 \leq n \leq 10^6 $
	\end{itemize}
\end{problem}

\begin{problem}{mulfuc}{mulfuc.in}{mulfunc.out}{1 second}{256}
	
	Mr. Hu 想让大家了解一下积性函数。
	
	积性是函数的一种重要性态,就像单调性、周期性一样。
	
	一个函数$f(n)$如果是积性的,当且仅当:
	$$
		f(nm) = f(n)f(m) \quad (gcd(m,n) = 1)
	$$
	
	如果$f(n)$是定义在$Z^{+}$上的积性函数,这样定义$Z^{+}$上的$g(n)$:
	
	$$
	g(n) = \sum_{d \mid n} f(d)
	$$
	
	那么可以证明$g(n)$也是一个积性函数。
	
	而在积性函数中,经常使用到以下几个重要的积性函数(容易证明它们都是积性的):
	
	\begin{itemize}
		\item $\tau(n)$  表示正整数$n$的正因子个数。
		\item $\sigma(n)$  表示正整数$n$的正因子和。
		\item $\mu(n)$  表示正整数$n$的Mobius函数值。
		\item $\varphi(n)$  表示正整数$n$的欧拉函数值,即$[1,n]$中与$n$互质的数的个数。
	\end{itemize}
	
	其中Mobius函数的定义如下: 
	
	$$
	\mu(n) = 
	\begin{cases}
	1,   & n = 1 \\
	(-1)^r  & n = p_1p_2\dots p_r \; (p_1 < p_2 < \dots < p_r) \; p_i \; is \; prime\\
	0  & other \; cases
	\end{cases}
	$$
	
	现在再定义两个函数:
	
	$$
	I(n) = \sum_{d \mid n} \mu(d)
	$$
	$$
	E(n) = \sum_{d \mid n} \varphi(d)
	$$
	
	现在Mr. Hu需要你去求出上面六个函数在$1 \leq n \leq 10^6$范围内的值。
	
	\InputFile 
	
	第$1$行包含一个整数$opt$,表示Mr. Hu需要你求出的函数的标号。对应关系是:
	
	\begin{table}[htbp]
		\centering
		\begin{tabular}{|c|c|c|c|c|c|c|}
			\hline
			$opt$ & $1$  & $2$ & $3$ & $4$ & $5$ & $6$ \\
			\hline
			函数 & $\tau$ & $\sigma$ & $\mu$ & $\varphi$ & $I$ & $E$ \\
			\hline
		\end{tabular}
	\end{table}
	
	\OutputFile
	
	输出$1$行,包含$opt$对应的函数在$[1,10^6]$范围内的函数值,两个函数值之间用一个空格隔开。
	
	\Example
	
	\begin{example}
		\exmp{
			1
		}{
		1 2 2 3 2 4 ...
	}%
\end{example}

\begin{example}
	\exmp{
		2
	}{
	1 3 4 7 6 12 ...
}%
\end{example}

\begin{example}
	\exmp{
		3
	}{
	1 -1 -1 0 -1 1 ...
}%
\end{example}

\begin{example}
	\exmp{
		4
	}{
	1 1 2 2 4 2 ...
}%
\end{example}

\Note

上面的样例输出只给出了前面几项,后面用省略号代替了,你需要输出全部的项。

本题满分$100$分,分为$6$个测试点,每个测试点$100/6$分

\begin{itemize}
	\item 对于第1个测试点,$opt = 1$。
	\item 对于第2个测试点,$opt = 2$。
	\item 对于第3个测试点,$opt = 3$。
	\item 对于第4个测试点,$opt = 4$。
	\item 对于第5个测试点,$opt = 5$。
	\item 对于第6个测试点,$opt = 6$。
\end{itemize}

\end{problem}

\end{document}
