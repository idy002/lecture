\documentclass{article}

\usepackage{amsmath}
\usepackage{amsfonts}
\usepackage{amssymb}

\usepackage{xeCJK}

\begin{document}
	\section{equation}
		\subsection{30\%数据}
			暴力,这时答案本身很小,所以只需要写一个dfs。
		\subsection{70\%数据}
			这个问题等价于“将$m$个相同的球放进$n$个不同的盒子中”的方案数,所以:
			
			$$
				ans \; = \; \binom{n+m-1}{m}
			$$
			
			(记得课上讲的夹棍法吗?)
			
			后者可以用递推公式算:
			$$
				\binom{n}{m} = \binom{n-1}{m-1} + \binom{n-1}{m}
			$$
			
			也可以暴力乘+逆元。
			$$
				\binom{n}{m} = \frac{n!}{m!(n-m)!}
			$$
			
			复杂度后者好点(因为模数不同,前者每次都要推一遍)
		\subsection{余下30\%数据}
			
			用公式推应该会超时,只能用后面那个,求逆元可以用扩展欧几里得,也可以欧拉定理(注意题目中$\varphi(pq) = (p-1)(q-1)$)。当然也可以在在模$p$和模$q$时答案求出来,再用中国剩余定理合并。
			
		\newpage
		
	\section{power}
		
		\subsection{10\%数据}
			暴力循环
			
		\subsection{30\%数据}
			普通二进制快速幂
			
		\subsection{70\%数据}
			高精度 + 二进制快速幂
			
		\subsection{100\%数据}
			十进制快速幂,类比二进制,从低到高维护好$3^{10^{i}}$。
		
		\newpage
		
	\section{comb}
		\subsection{30\%数据}
			暴力算出组合数,用递推公式
			
		\subsection{100\%数据}
			题目就是求:
			$$
				p \mid \binom{n}{i}
			$$
			即:
			$$
				\binom{n}{i} \equiv 0 \; ( mod \; p )
			$$
			的$i$的个数。想到可能和Lucas有关,我们先把问题转化成求:
			$$
				p \nmid \binom{n}{i}
			$$
			的$i$的个数。
			由Lucas定理:
			$$
				\binom{n}{m} \equiv \binom{n_1}{m_1}\binom{n_2}{m_2}\cdots\binom{n_s}{m_s}  \; ( mod \; p )
			$$
			其中$n_1n_2\dots n_s$和$m_1m_2\dots m_s$是$n$和$m$的$p$进制分解。
			容易发现:只要右边每一项模意义下都非$0$,那么右边乘起来都非零,这样的数$m$有:
			$$
				(n_1+1)(n_2+1)\cdots(n_s+1)
			$$
			个,这就是不满足条件的数的个数,那么答案就是:
			$$
				(n + 1) - (n_1+1)(n_2+1)\dots(n_s+1)
			$$
			\newpage
	\section{derange}
		\subsection{30\%数据}
			用\verb|next_permutation|枚举排列,然后暴力check。
		
		\subsection{70\%数据}
			
			我们用$D[n]$表示$n$个元素的错排数量,那么我们可以通过枚举不合法的排列重合的个数来算出$D[n]$:
			$$
				D[n] = n! - \sum_{i = 1}^{n} \binom{n}{i}D[n-i]
			$$
			初始值是:
			$$
				D[0] = 1
			$$
			我们可以先处理出组合数,再跑这个递推,就行了。
			
		\subsection{100\%数据}
			$D[n]$中,我们先选择$n$号元素的位置,有$n-1$种选择($[1,n-1$]) ,假设$n$放在了$i$号位置,我们再考虑$i$号元素放哪,此时分两种情况:
			\begin{itemize}
				\item $i$放在$n$号位置,此时有$D[n-2]$种放法。
				\item $i$号元素不放在$n$号位置,问题就转化成了:我要将$[1,n-1]$这$n-1$个元素放进$[1,i-1], [i+1,n]$这些位置,并且每个元素都有且仅有一个位置不能放($1$不能放在$1$,$2$不能放在$2$,\dots $i$不能放在$n$ \dots )。这不就是$n-1$个元素的错排吗?(将$n$号位置看成$i$号元素本来的位置)。此时有$D[n-1]$种放法。
			\end{itemize}
		\newpage
	\section{mulfunc}
		我们先证明一下题目中给出的结论吧:
		$$
			g(n) = \sum_{d \mid n}f(d)
		$$
		假设$n$有质因分解:$n = p_1^{\alpha_1}p_2^{\alpha_2}\dots p_s^{\alpha_s}$。从而:
		$$
			g(p_1^{\alpha_1}p_2^{\alpha_2}\dots p_s^{\alpha_s}) = \sum_{d \mid p_1^{\alpha_1}p_2^{\alpha_2}\dots p_s^{\alpha_s}}f(d) = \sum_{i_1 = 0}^{\alpha_1}\sum_{i_1 = 0}^{\alpha_1}\cdots\sum_{i_s = 0}^{\alpha_s}f(p_1^{i_1}p_2^{i_2}\dots p_s^{i_s}) 
		$$
		因为$f(n)$是积性函数,我们可以继续:
		$$
			= \sum_{i_1 = 0}^{\alpha_1}\sum_{i_1 = 0}^{\alpha_1}\cdots\sum_{i_s = 0}^{\alpha_s}f(p_1^{i_1})f(p_2^{i_2})\dots f(p_s^{i_s}) 
		$$
		由求和的性质,又有:
		$$
			= \sum_{i_1 = 0}^{\alpha_1}f(p_1^{i_1})\sum_{i_1 = 0}^{\alpha_1}f(p_2^{i_2})\cdots\sum_{i_s = 0}^{\alpha_s}f(p_s^{i_s}) =  (\sum_{i_1 = 0}^{\alpha_1}f(p_1^{i_1}))(\sum_{i_1 = 0}^{\alpha_1}f(p_2^{i_2}))\cdots(\sum_{i_s = 0}^{\alpha_s}f(p_s^{i_s}))
		$$
		$$
			= g( p_1^{\alpha_1})g(p_2^{\alpha_2})\dots g(p_s^{\alpha_s})
		$$
		从而有:
		$$
		g(p_1^{\alpha_1}p_2^{\alpha_2}\dots p_s^{\alpha_s}) = g( p_1^{\alpha_1})g(p_2^{\alpha_2})\dots g(p_s^{\alpha_s})
		$$
		就证明了$g(n)$的积性。
		
		所以,关于$\tau(n)$和$\sigma(n)$又有下面的定义:
		$$
			\begin{array}{rcl}
				\tau(n) &=& \sum_{d \mid n}1	\\
				\sigma(n) &=& \sum_{d \mid n}d
			\end{array} 
		$$

		显然$f(n)=1$和$g(n)=n$都是积性函数,所以$\tau(n)$和$\sigma(n)$是积性函数。
		
		由上面的结论,我们想要求一个积性函数的值,只需要弄清楚$f(p^\alpha)$的值就行了。
		
		$$
			\begin{array}{ccl}
				\tau(p^\alpha) &=&\alpha + 1	\\
				\sigma(p^\alpha)  &=& 1 + p + p^2 + \cdots + p^\alpha = \frac{p^{\alpha+1}-1}{(p-1)}	\\
				\mu(p) &=& -1	\\
				\mu(p^\alpha) &=& 0 \quad ( \alpha \geq 2 )\\
				\phi(p^\alpha) &=& p^\alpha - p^{\alpha - 1}\\
			\end{array}
		$$
		因为我们线性筛的时候会找到每个数的最小素因子,所以可以在线性筛是算出所有数的函数值,具体参见代码。
		
	
\end{document}