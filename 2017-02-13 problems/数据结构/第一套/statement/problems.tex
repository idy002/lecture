\documentclass[11pt,a4paper,oneside]{article}
\usepackage[english]{babel}
\usepackage{olymp}
\usepackage[dvips]{graphicx}
\usepackage{color}
\usepackage{colortbl}
%\usepackage{expdlist}
%\usepackage{mfpic}
%\usepackage{comment}
\usepackage{multirow}
\usepackage{amsmath}
\usepackage{amsfonts}
\usepackage{amssymb}
\usepackage{xeCJK}

%\setCJKmainfont[BoldFont={Hei}]
%{SimSun}
%\setCJKmonofont{FangSong}

\renewcommand{\contestname}{
No.7 High School Data Structure Special Training 1\\
idy002, \today
}    

\begin{document}
	
\begin{problem}{rotinv}{rotinv.in}{rotinv.out}{2 seconds}{256}
	
	如果你有一个长度为$n$的序列:
	
	$$a_1,a_2,a_3,\dots,a_n$$
	
	那么它的一个逆序对是一个二元组:$<i,j>$满足$i<j$且$a_i > a_j$,其中$i, j \in [1,n]$。
	
	我们称一个序列所包含的逆序对的个数为这个序列的逆序对数。
	
	那么问题来了:
	
	我给出一个长度为$n$的序列,需要你计算:
	$$
		\begin{array}{c c c c c}
			a_1,&a_2&\dots&a_{n-1},&a_n	\\
			a_2,& a_3& \dots& a_{n},   & a_1	\\
			a_3,& a_4& \dots& a_1,       & a_2	\\
			& & \cdots 	& & \\
			a_n,& a_1& \dots& a_{n-2},& a_{n-1}	\\
		\end{array}
	$$
	
	这$n$个序列的逆序对之和。

	\InputFile
	输入文件包含$2$行:
	
	第$1$行$1$个整数:$n$,表示给定序列的长度。
	
	第$2$行$n$个整数:$a_1 \; a_2 \; \dots a_n $,表示初始序列。
	
	\OutputFile
	
	输出$n$个序列的逆序对的和。
	
	\Example
	
	\begin{example}
		\exmp{
			3
			2 2 3
		}{
			6
	}%
	\end{example}
	
	以上样例中,$3$个序列分别是:$2 \; 2 \; 3 $,$2 \; 3 \; 2 $,$3 \; 2 \; 2 $,分别有$0$,$1$,$2$个逆序对,所以和为$6$。
	
	\begin{example}
		\exmp{
			3
			1 1 1
		}{
			0
		}%
	\end{example}
	
	以上样例中,$3$个序列都是:$ 1 \; 1 \; 1 $,逆序对数为$0$,所以答案为$0$。
	
	\Note
	\begin{itemize}
		\item 对于$30\%$的数据,$1 \leq n \leq 300$
		\item 对于$60\%$的数据,$1 \leq n \leq 5000$
		\item 对于$100\%$的数据,$1 \leq n \leq 10^6$,$ 1 \leq a_i \leq n$
	\end{itemize}

\end{problem}

\begin{problem}{rise}{rise.in}{rise.out}{2 seconds}{256}
	
	你有一堆柱子,它们竖直地并排摆放在桌子上,它们的高度分别是:
	
	$$
		h_1, h_2, h_3, \dots, h_n
	$$
	
	你从前往后看,能够看见的柱子个数为这个柱子序列的“可见度”(能够看见柱子$i$当且仅当$h_j < h_i \; \forall   j < i$)。
	
	现在给你一个长度为$n$的序列,还有$m$个询问,每次询问某个区间$[l,r]$的柱子单独拿出来后,其可见度是多大。
	
	\InputFile
	
	第$1$行$2$个整数:$n \; m$,表示给出的柱子序列的长度和询问数。
	
	第$2$行$n$个整数:$a_1 \; a_2 \; a_3 \; \dots \; a_n$,表示每根柱子对应的高度。
	
	接下来$m$行,每行$2$个整数:$l \; r$,表示对区间$[l,r]$进行询问。
	
	\OutputFile

	对于每个询问,输出答案。

	\Example
	
	\begin{example}
		\exmp{
			5 4
			1 3 2 4 2
			1 4
			2 4
			1 3
			2 3
		}{
			3
			2
			2
			1
		}%
\end{example}

	样例中“能够看见”的柱子的高度分别是:$1 \; 3 \; 4$,$ 3 \; 4$,$1 \; 3 $,$ 3 $
	
\Note
	\begin{itemize}
		\item 对于$30\%$的数据,$ 1 \leq n, m \leq 10^3 $
		\item 对于$100\%$的数据,$ 1 \leq n, m \leq 10^5 $,$1 \leq a_i \leq n$,$ 1 \leq l \leq r \leq n $
	\end{itemize}
	
\end{problem}

\begin{problem}{seqmod}{seqmod.in}{seqmod.out}{2 seconds}{256}
	
	给你一棵无根树,边有边权,且是$[0,9]$之间的整数,给你$m$个询问,每次询问两个点$u, v$之间的路径的边的边权顺次连接起来后构成的那个数字取模于$31$。

	\InputFile
	
	第$1$行$2$个整数:$n \; m$,表示树的节点个数和询问数。
	
	接下来$n-1$行,每行$3$个数:$u \; v \; d$表示点$u$和点$v$之间有一条边权为$d$的边。
	
	接下来$m$行,每行$2$个整数:$u \; v$表示一个询问。
	
	\OutputFile
	
	对于每个询问输出其答案。
	
	\Example
	
	\begin{example}
		\exmp{
			5 3
			1 2 2
			1 3 8
			3 4 9
			3 5 2
			1 4
			2 5
			5 2
		}{
			27
			24
			6
		}%
	\end{example}
	
	样例中三条路径对应的数字分别是:$89$,$582$,$285$,它们被$31$取模后为:$27$,$24$,$6$。

\Note
\begin{itemize}
	\item 对于$30\%$的数据,$ 1 \leq n, m \leq 10^3 $
	\item 对于$100\%$的数据,$ 1 \leq n, m \leq 10^5$,$ 1 \leq u, v \leq n $,$ u \neq v$,$ 0 \leq d \leq 9$
\end{itemize}

\end{problem}

\end{document}
