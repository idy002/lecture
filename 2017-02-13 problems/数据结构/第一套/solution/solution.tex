\documentclass{article}

\usepackage{amsmath}
\usepackage{amsfonts}
\usepackage{amssymb}

\usepackage{xeCJK}

\begin{document}
	\section{rotinv}
		\subsection{30\%}
		
			$O(n^3)$暴力:枚举每个循环状态,再暴力计算逆序对数。
			
		\subsection{60\%}
			法一:
			
			将原序列复制一份接在自己后面,形成一个长度为$2n$的序列,先暴力计算$[1,n]$的逆序对数,然后将这个区间向右边移动,维护答案。具体来说,就是假如我们算出了$[i,j]$中的逆序对数,我们怎么算$[i+1,j+1]$呢,其实就是加上$[i+1,j]$中比$a[j+1]$大的数的个数再减去$[i+1,j]$中比$a[i]$小的数的个数。所以我们可以暴力算那两个数量,连续$n$个长度为$n$的子区间的逆序数的和就是答案,复杂度$O(n^2)$。
			
			法二:
			
			计算一个长度为$n$的序列的逆序对的个数可以通过数据结构(树状数组或线段树)优化到$O(nlogn)$,维护$[1,i-1]$的一个值的分布情况(加入到线段树中),每次查询这些数比$a[i]$大的有多少个,再将$a[i]$加入到线段树中去。所以计算$n$个序列的逆序数只需要$O(n^2logn)$.
			
		\subsection{100\%}
			
			将上面的法一和法二结合一下,用线段树优化法一中那个暴力,使得可以$O(logn)$查询,最后复杂度$O(nlogn)$.
		
		\newpage
		
	\section{rise}
		\subsection{30\%}
		
		暴力$O(n^2)$.
		
		\subsection{100\%}
		
		用线段树解决,每个节点维护:
		
		\begin{itemize}
			\item 最大值:vmax
			\item 这个区间对应的答案:c1
			\item 这个区间的右儿子(如果有的话)去掉小于等于左儿子最大值以后的剩下的那些数对应的答案:c2
		\end{itemize}
		
		每个节点支持一种询问query(nd,h):询问去掉nd这个节点中小于等于h的数之后这个区间对应的答案.
		
		假如我们可以实现上面这个操作的话,我们先将询问区间$[L,R]$对应的那些节点提取出来排好:nd1, nd2, nd3, ... nds.
		
		query(nd1,0) + query( nd2, max(nd1)) + query(nd3,max(nd1,nd2)) + ...就是答案.
		
		我们考虑怎样实现这个询问过程query(nd,h):\\
			\quad  如果是叶子节点,只需要返回[h < a]	 \\
			\quad  如果h < nd左儿子最大值,则返回 query(nd左儿子,h) + 右儿子c2	\\
			\quad  如果h >= nd,则返回query(nd右儿子,h)	\\

		其实可以将query看成我用h去砍一刀,然后计算答案,上面的过程的正确性还是比较容易理解的,如果我还有什么没有说清楚的可以看一下代码.
		
		容易发现query每次最多选两个儿子中的一个节点走下去,所以最多走$O(logn)$个节点,所以复杂度是$O(logn)$的.
		
		每个节点在算c2时需要调用一次query,所以build的时间复杂度是$O(nlogn)$的.
		
		询问的时候,最多提取出来$O(logn)$个区间,每个区间进行一次$O(logn)$的询问,所以单次询问是$O(log^2n)$的.
		
		总的复杂度是$O(nlogn+mlog^2n)$的.
		
		\newpage 
		
	\section{seqmod}
		\subsection{30\%}
		
		随便怎么暴力都可以...
		
		\subsection{100\%}
		
		就是有序链剖,细节我课上讲过了,大家看一下代码可以看懂的.与普通链剖的区别就是这个东西要在乎顺序,所以不能用swap那种写法,两个各跳各的,还要注意合并的顺序.
		
\end{document}