\documentclass[11pt,a4paper,oneside]{article}
\usepackage[english]{babel}
\usepackage{olymp}
\usepackage[dvips]{graphicx}
\usepackage{color}
\usepackage{colortbl}
%\usepackage{expdlist}
%\usepackage{mfpic}
%\usepackage{comment}
\usepackage{multirow}
\usepackage{amsmath}
\usepackage{amsfonts}
\usepackage{amssymb}
\usepackage{xeCJK}

%\setCJKmainfont[BoldFont={Hei}]
%{SimSun}
%\setCJKmonofont{FangSong}

\renewcommand{\contestname}{
No.7 High School Data Structure Special Training 5\\
idy002, \today
}    

\begin{document}
\begin{problem}{setmod}{setmod.in}{setmod.out}{2 seconds}{256}
		
	给你一个序列:$a_1 \; a_2 \; a_3 \dots  a_n $,有$m$个操作,操作如下:
	
	\begin{itemize}
		\item \texttt{modify l r x} 将区间$[l,r]$中的每个数修改为$x$
		\item \texttt{change l r x} 将区间$[l,r]$中的每个数加上$x$
		\item \texttt{query l r } 询问区间$[l,r]$中的和
	\end{itemize}
	
	\InputFile
	
	第$1$行$2$个整数:$n \; m$,表示序列长的和操作数.

	第$2$行$n$个整数:$a_1 \; a_2 \; a_3 \dots a_n$,表示初始序列.
	
	接下来$m$行,每行是上面三种操作中的一种.
	
	\OutputFile
	
	对于每个询问操作,输出其结果.
	
	\Example
	
	\begin{example}
		\exmp{
			3 3
			1 2 3
			change 1 3 2
			modify 3 3 3
			query 1 3
		}{
			10
		}%
	\end{example}
	
	\Note
	\begin{itemize}
		\item 对于$30\%$的数据,$1 \leq n, m \leq 10^3$
		\item 对于$100\%$的数据,$1 \leq n, m \leq 10^5$,$1 \leq a_i, x \leq n$,$1 \leq l \leq r \leq n$
	\end{itemize}
	
\end{problem}
	\begin{problem}{area}{area.in}{area.out}{2 seconds}{256}
		
	给出$n$个矩形,求它们的面积并.
	
	更准确一点,每个矩形将给出它的左上角和右下角的位置:$x_1, y_1, x_2, y_2$
	
	这四个数都是整数且满足$x_1 \leq x_2, y_1 \leq y_2$.
	
	我们需要你求:
	
	$$area = \mid \{ (x,y) \in Z\times Z \mid \exists \; a \; rect. \; s.t. \; x_1 \leq x \leq x_2 \; and \; y_1 \leq y \leq y_2  \} \mid $$
		
	\InputFile
	
	第$1$行$1$个整数:$n$,表示矩形的个数。
	
	接下来$n$行,每行$4$个整数:$ x_1\; y_1 \; x_2 \; y_2 $,表示一个矩形的左上角和右下角的坐标。
	
	\OutputFile
	
	输出$area$。
	
	\Example
		
	\begin{example}
		\exmp{
			3
			1 1 2 3
			1 2 3 3
			3 3 4 4
		}{
			11
		}%
	\end{example}
	
	样例解释:一共有$11$个点落在了上面三个矩形所表示的区域内:
	
	$(1,1), (1,2), (1,3), (2,1), (2,2), (2,3), (3,2), (3,3), (3,4), (4,3), (4,4)$
	
	\Note
	\begin{itemize}
		\item 对于$30\%$的数据,$1 \leq n \leq 100$, $1 \leq x_1 \leq x_2 \leq 100$, $1 \leq y_1 \leq y_2 \leq 100$
		\item 对于$100\%$的数据,$1 \leq n \leq 10^5$, $1 \leq x_1 \leq x_2 \leq 10^5$, $ 1 \leq y_1 \leq y_2 \leq 10^5$
	\end{itemize}
	
\end{problem}

\begin{problem}{intkth}{intkth.in}{intkth.out}{3 seconds}{512}
	
	我看好你哟。
	
	给你一个长度为$n$的序列,有$m$个操作:
	
	\begin{itemize}
		\item \texttt{modify u x} 将第$u$个数修改为$x$
		\item \texttt{query l r k} 询问区间$[l,r]$中第$k$小的数\footnote{第$k$小是指将区间$[l,r]$从小到大排序后,第$k$个数}
	\end{itemize}
	
	\InputFile

	第$1$行$2$个整数:$n \; m$,表示序列长度和操作数。
	
	第$2$行$n$个整数:$a_1 \; a_2 \; a_3 \; \dots \; a_n$,表示给定序列。
	
	接下来$m$行,每行表示上面的某个操作。

	\OutputFile
	
	对于每个询问操作,输出其结果。
	
	\Example
	
	\begin{example}
		\exmp{
			5 5
			5 2 1 3 4 
			query 1 4 3
			modify 4 5
			query 1 4 3
			modify 1 3
			query 1 4 3
		}{
			3
			5
			3
		}%
	\end{example}

	\Note
	\begin{itemize}
		\item 对于$30\%$的数据,$1 \leq n, m \leq 10^3$
		\item 对于$100\%$的数据,$1 \leq n, m \leq 10^5$,$1 \leq a_i, u, x \leq n$, $ 1 \leq k \leq r - l + 1$, $1 \leq l \leq r \leq n$ 
	\end{itemize}

\end{problem}

\end{document}
