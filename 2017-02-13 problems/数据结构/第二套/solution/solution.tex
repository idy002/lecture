\documentclass{article}

\usepackage{amsmath}
\usepackage{amsfonts}
\usepackage{amssymb}

\usepackage{xeCJK}

\begin{document}
	\section{setmod}
		\subsection{30\%}
			$O(nm)$暴力。
		\subsection{100\%}
			要维护几个东西:
			\begin{itemize}
				\item \texttt{sum} 表示区间的和
				\item \texttt{type} 表示现在的标记类型(可以是没有标记,可以是增量标记,可以是赋值标记)
				\item \texttt{delta} 如果是增量标记,那么这个里面存的增量
				\item \texttt{value} 如果是赋值标记,那么这里面就存的是那个值
			\end{itemize}
			然后就秩序要讨论一下发现:
			\begin{itemize}
				\item 空标记 + 赋值操作 = 赋值标记
				\item 空标记 + 增量操作 = 增量标记
				\item 增量标记+赋值操作 = 赋值标记
				\item 增量标记+增量操作 = 增量标记
				\item 赋值标记+增量操作 = 赋值标记
				\item 赋值标记+赋值操作 = 赋值标记
			\end{itemize}
			因为它们之间能和谐共处(如果A标记遇见B操作无法转化为一种现有的标记,那么他们就不能同时用线段树实现),所以就能完成两种操作同时进行。具体看代码是怎样合并的。
			
			\newpage
			
	\section{area}
		\subsection{30\%}
			枚举点,判断它是否在某个矩形中。$O(n^3)$
		\subsection{100\%}
			扫描线,对于一个举行(x1,y1,x2,y2),将它看成两个事件:在x1这个时间将(y1,y2)这个区间加一,在x2+1这个时间将(y1,y2)这个区间减一。
			
			这样,我们遍历整个时间,并在执行完这个时间的操作后看看有多少位置非0,将其数量加到答案中,就完了,当然时间不能傻傻地一个一个枚,因为关键的时间点最多2n个,其它时候面积是没有变的,所以要一段一段地算。
			
			至于怎么用线段树实现那么查看有多少个非零的位置,需要注意对于任何一个减一操作,前面一定有一个和它一样的加一操作,就只需要维护一下每个节点被完全覆盖的次数。再用另一个来统计子树中的那些修改导致这个节点还有多少个非零。有点像标记永久化(我们讲的第二种区间修改的写法)。
			
						
						\newpage
						
	\section{intkth}
		\subsection{30\%}
			暴力,sort,$O(mnlogn)$
		\subsection{100\%}
			树状数组套可持久化值域线段树,每次询问就是对$O(logn)$棵树加加减减后求第$k$大。修改也会同时修改$O(logn)$棵树,所以总的复杂度是:$O((m+n)log^2n)$。这道题我记得课上讲过的哟。
			
						
			\newpage
						
\end{document}