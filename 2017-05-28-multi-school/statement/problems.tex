\documentclass[11pt,a4paper,oneside]{article}
\usepackage[english]{babel}
\usepackage{olymp}
\usepackage[dvips]{graphicx}
\usepackage{color}
\usepackage{colortbl}
%\usepackage{expdlist}
%\usepackage{mfpic}
%\usepackage{comment}
\usepackage{multirow}
\usepackage{amsmath}
\usepackage{amsfonts}
\usepackage{amssymb}
\usepackage{xeCJK}

%\setCJKmainfont[BoldFont={Hei}]
%{SimSun}
%\setCJKmonofont{FangSong}

\renewcommand{\contestname}{
No.7 High School Number Theory Special Training \\
idy002, \today
}    

\begin{document}

\begin{problem}{gcount}{gcount.in}{gcount.out}{1 second}{256}
	
	这是一个多雨的时代。
	
	Darrell所在的国家有$N$个城镇,并且有$N-1$条输水管道连接它们,使得任意两个城镇都可以通过一系列的输水管道连接起来。
	
	如果城镇$u$爆发了洪水,洪水会向四周蔓延(通过输水管道),但由于不知道水蔓延的速度,也不知道水会蔓延多久,从而导致不清楚洪水过后,城镇的受灾情况(即有哪些城镇受到影响,只要洪水流到该城市该城市就会受影响)。
	
	现在,国家的城市防洪委员会想要知道,如果城市$u$爆发洪水,有多少种不同的受灾情况,以便书写救灾预案。你需要回答$u$在$[1,N]$内的答案。但答案可能过大,你只需要将它对$10^{9}+7$取模。
	
	\InputFile
	
	第$1$行,有$1$个整数:$N$,表示城镇数。
	
	接下来$N-1$行,每行$2$个整数:$u \; v$,表示$u$和$v$之间有输水管道连接。

	\OutputFile
	
	输出$1$行,包含$N$个数,第$i$个数表示如果城市$i$遭遇洪水,则受影响的情况可能有多少种。(取模后输出)

	\Example
	
	\begin{example}
		\exmp{
			4
			1 2
			2 3
			2 4
		}{
		    5 8 5 5
		}%
\end{example}

\Note

	样例中,$1$号城市发洪水后,可能的受灾情况为:$[1], [1,2], [1,2,3], [1,2,4], [1,2,3,4]$。
	
	
		
	\begin{itemize}
		\item a
	\end{itemize}
\end{problem}

\end{document}
