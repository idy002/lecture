\documentclass{beamer}
\usepackage{xeCJK}
\usepackage{listings}
\usepackage{graphicx}

\title{数据结构第二讲}
\subtitle{一些关于树的统计问题}
\author{丁尧尧}
\institute{上海交通大学}
\date{\today}
\usetheme{CambridgeUS}

\begin{document}
	\maketitle
	\begin{frame}{目录}
		\tableofcontents
	\end{frame}
	\section{一些关于树的统计问题}
		\begin{frame}{关于树的一些统计问题}
			算法竞赛中,我们经常遇到一些和树有关的统计问题(以点、链、子树作为修改和询问对象)。\\
			这里列出一些基本的问题:\\
			\begin{enumerate}
				\item 单点修改,子树询问
				\item 子树修改,单点询问
				\item 子树修改,子树询问
				\item 单点修改,链询问
				\item 链修改,单点查询
				\item 链修改,子树修改,链查询,子树查询
			\end{enumerate}
			上面最后一个是前五个的更一般的情况(点是特殊的链),我们留到后面解决,今天主要解决前面的问题。要解决前面5个问题,我们需要学习一下最近公共祖先和dfs序这两个概念。
		\end{frame}
	\section{DFS序}
		\subsection{概念}
			\begin{frame}{DFS序}
				DFS序有三种,我们姑且用序列的长度来对它们进行分类:
				\begin{itemize}
					\item $n$个点,这是最简单、最常用的一种
					\item $2*n$个点,这个用得不多,但还是有些有用的性质
					\item $2*n - 1$个点,这个主要是用于$O(1)$求lca  
				\end{itemize}
			\end{frame}
			\begin{frame}[fragile=singleslide]
				\frametitle{$n$个点}
				先看看我们怎么得到:\\
				\begin{verbatim}
					int in[N], out[N], seq[N], idc;
					void dfs( int u, int f ) {
					    seq[++idc] = u;
					    in[u] = idc;
					    for( int t = head[u]; t; t = last[t] ) {
					        int v = dest[t];
					        if( v == f ) continue;
					        dfs( v, u );
					    }
					    out[u] = idc;
					}
					idc = 0;
					dfs( root, root );
				\end{verbatim}
			\end{frame}
			\begin{frame}
				这种DFS序满足:
				\begin{itemize}
					\item 每个节点和DFS序中的位置一一对应,即\texttt{u}所在位置是\texttt{in[u]},位置\texttt{i}对应节点是\texttt{seq[i]}。
					\item 每棵子树在DFS序中是连续的,即\texttt{u}节点代表的子树的所有节点都在\texttt{[in[u],out[u]]}中。
				\end{itemize}
				有了上面的两个性质,子树修改就是DFS序上的区间修改,子树询问就是DFS序上的区间询问,从而可以用树状数组或线段树解决前三个问题。
			\end{frame}
			\begin{frame}{2*n个点}
				这个又称树对应的一个括号序列。
				与上面不同的是,我们不光进来的时候将点加到序列中,出去的时候也加一次。\\ 
				如果我们把进来的时候那次改成左括号,出去的那次看成右括号,那么我们得到的就是一个层层嵌套的括号序列。 \\
				\texttt{[1,in[u]]}中那些没有被其有括号匹配的左括号都是从根节点到\texttt{u}的点对应的左括号。\\
				\texttt{[in[u]+1,in[v]}中那些匹配剩下的括号个数就是\texttt{u}到\texttt{v}的节点数。
			\end{frame} 
			\begin{frame}{2*n-1个点}
				DFS的过程实际上也是一个遍历边的过程,我们把边顺次接起来,那么那些串起来的节点按照顺序排列起来就是这种DFS序,有$2m + 1 = 2n - 1$个节点。\\
				如果我们用\texttt{in[u]}表示序列中\texttt{u}节点第一次出现的位置,那么\texttt{[in[u],in[v]}中深度最小的节点就是\texttt{u}和\texttt{v}的最近公共祖先(下面讲)。
			\end{frame} 
		\subsection{应用}
	\section{最近公共祖先}
		\subsection{问题}
			\begin{frame}{最近公共祖先}
				\begin{definition}{最近公共祖先}
					树$T$中,如果\texttt{a}在\texttt{u}到根的路径上,我们称\texttt{a}是\texttt{u}的祖先。\\
					如果\texttt{a}既是\texttt{u}的祖先,又是\texttt{v}的祖先,并且其深度是所有满足条件的点中最大的,我们称其为\texttt{u}、\texttt{v}的最近公共祖先。
				\end{definition}
				求最近公共祖先是一个很基本的问题,许多其他的算法都需要快速地求出任意两个点的最近公共祖先。
			\end{frame}
		\subsection{朴素做法}
			\begin{frame}[fragile=singleslide]
				\frametitle{朴素做法}
				我们可以dfs一遍这颗树,把每个点的深度和父亲搞出来,然后每次将\texttt{u}和\texttt{v}中深度较大的那个点跳到它父亲,直到两个点重合:
				\begin{verbatim}
				whie( u != v ) {
				    if( dep[u] > dep[v] ) 
				        u = fat[u];
				    else
				        v = fat[v];
				}
				return u;
				\end{verbatim} 
				但这样最坏是$O(n)$复杂度的。
			\end{frame}
		\subsection{倍增做法}
			\begin{frame}{倍增求LCA}
				我们上一讲学习了ST表,这次我们类似地定义:\\
				\begin{center}
					\texttt{int anc[N][P+1];}	\\
				\end{center}
				\texttt{anc[u][p]}表示\texttt{u}节点向上“跳”了$2^p$步之后到达的节点(我们可以认为根节点向上跳一步又跳回了自己)。\\
				和ST表一样,我们可以用$O(nlogn)$的时间将\texttt{anc}数组求出来。\\
				然后算法的大概思路是:
				\begin{enumerate}
					\item 让深度较深的点向上跳,直到两个点的深度相同
					\item 让两个点一起跳,跳到它们的lca
				\end{enumerate}
			\end{frame}
		\subsection{RMQ做法}
			\begin{frame}{RMQ做法}
				上面单次查询是$O(log(n))$的,其实还有查询更块的算法($O(1)$查询)。就是用上面提到的DFS序。
				有了第三种DFS序的性质之后,我们可以\textbf{以深度为比较关键字},建立ST表,求出\texttt{[in[u],in[v]]}中深度最小的节点,其即为我们要求的\texttt{lca(u,v)}。
			\end{frame}
		\subsection{后两个问题}
			\begin{frame}{后两个问题}
				有了lca以后,后两个问题我们也能解决了。但是要转换一下。\\
				我们把问题特殊化一下,加入我们只修改或询问从根节点开始的一条链,该怎么做?\\
				对于单点修改,链查询,我们单点修改时,修改整个子树,查询时查询单点。\\
				对于链修改,单点查询,我们链修改时修改单点,查询时查询子树。\\
				让后如果问题满足“相减性”,我们可以用关于\texttt{u, v, lca(u,v), fa[lca(u,v)]}这几个点到根节点的链来拼凑出\texttt{u,v}之间的链。
			\end{frame}
\end{document} 

