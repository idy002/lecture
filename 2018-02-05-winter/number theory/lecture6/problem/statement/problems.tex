\documentclass[11pt,a4paper,oneside]{article}
\usepackage[english]{babel}
\usepackage{olymp}
\usepackage[dvips]{graphicx}
\usepackage{color}
\usepackage{ulem}
\usepackage{colortbl}
%\usepackage{expdlist}
%\usepackage{mfpic}
%\usepackage{comment}
\usepackage{multirow}

\usepackage{xeCJK}

%\setCJKmainfont[BoldFont={Hei}]
%{SimSun}
%\setCJKmonofont{FangSong}

\renewcommand{\contestname}{
No.7 High School Winter Training - Number Theory \\
idy002, \today
}    

\begin{document}

\begin{problem}{modeq}{modeq.in}{modeq.out}{1 second} 

	我们来个解方程完整版。
	
    \InputFile
	第一行一个整数$ T $,表示需要求解的方程数。
	
	接下来$ T $行,每行三个整数:$ a, b, c $表示一个方程:
	$$
		ax + by = c
	$$

    \OutputFile
	对于每个方程:
	\begin{itemize}
		\item 如果方程无解,输出$No$
		\item 如果方程有解,输出四个整数:$x_1 \; y_1 \;  x_2 \; y_2$,表示两组解,其中$x_1 \; y_1$ 表示$x$是最小非负时对应的解,其中$x_2 \; y_2$表示$y$是最小非负时的解。\footnote{即不存在解:$x_0 \; y_0$,使得 $0 \leq x_0 < x_1$ 或者 $ 0 \leq y_0 < y_1$}
	\end{itemize}

    \Example

    \begin{example}
        \exmp{
			3
			4 6 3
			3 4 7
			-2 3 6
        }{
			No
			1 1 1 1
			0 2 -3 0
        }%
    \end{example}

    \Note
    
    \begin{itemize}
		\item 对于$100\%$的数据,$0 < \mid a \mid, \mid b \mid, \mid c \mid \leq 10^9 $, $ 1 \leq T \leq 1000$
    \end{itemize}

\end{problem}

\begin{problem}{crt}{crt.in}{crt.out}{5 second} 
	
	\sout{我们来个孙子定理完整版。}
	
	可能现在还来不起。。。我们就先假设$m_i$两两互质吧。
	
	\InputFile
	第一行一个整数$ T $,表示需要求解的方程组数。
	
	接下来$ T $个方程组,对于每个方程组:
	
	第$1$行一个整数$n$,表示方程组对应的方程个数。
	
	接下来$n$行,第$i$行两个数:$a_i \; m_i$表示第$i$个方程:
	$$
		x \equiv a_i \; ( mod \; m_i )
	$$
	
	\OutputFile
	对于每个方程:输出最小非负整数解。
	
	\Example
	
	\begin{example}
		\exmp{
			3
			2
			3 5
			4 7
		}{
			18
	}%
\end{example}

\Note

\begin{itemize}
	\item 对于$100\%$的数据,$ 0 \leq a_i < m_i \leq 100 $,$ 1 \leq n \leq 4$,$1 \leq T \leq 100000$,保证$m_i$两两互质。
\end{itemize}

\end{problem}

\begin{problem}{seq}{seq.in}{seq.out}{1 second} 
	
	我们来个递推精简版。求下面数列的第$n$项:
	$$
		f(0) = a_0, f(1) = a_1, f(2) = a_2
	$$
	$$
		f(n) = bf(n-1) + cf(n-2) + df(n-3) + e \quad (n \geq 3)
	$$
	
	\InputFile
	包含$1$行,共$8$个整数:$ a_0 \; a_1 \; a_2 \; b \; c \; d \; e \; n $。
	
	\OutputFile
	输出$f(n)$的后$18$位(后$18$位的前缀$0$需要输出,不足$18$位用$0$补齐)。
	
	\Example
	
	\begin{example}
		\exmp{
			1 2 3 4 5 6 7 3 
		}{
			 000000000000000035
		}%
\end{example}

\Note

\begin{itemize}
	\item 对于$30\%$的数据,$ 0 \leq a_0, a_1, a_2, b, c, d, e, n \leq 10^{6}$
	\item 对于$100\%$的数据,$ 0 \leq a_0, a_1, a_2, b, c, d, e, n \leq 10^{18}$
\end{itemize}

\end{problem}



\end{document}
