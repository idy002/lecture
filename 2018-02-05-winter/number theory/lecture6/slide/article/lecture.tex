\documentclass{article}
\usepackage{amsthm}
\usepackage{amssymb}
\usepackage{xeCJK}
\usepackage{listings}
\usepackage{graphicx}
\usepackage{multicol}

\newtheorem{theorem}{定理}
\newtheorem{definition}{定义}
\newtheorem{example}{例}
\newtheorem{prop}{命题}

\title{数论选讲}
\author{丁尧尧}
\date{\today}



\begin{document}
	\maketitle
	
	\newpage

	\tableofcontents

	\newpage	
	
	\begin{center}
		本文梳理了一下信息学竞赛中常用的数论知识, 目的在于让大家快速理解入门, 有些地方如果自然思维能很快感觉到其正确性, 就没有深究其细节.
	\end{center}
	
	\newpage
	
	\section{基础内容}
		以下内容,如果不特别说明,都是在整数范围内讨论.
	\subsection{整除}

		\begin{theorem}
			如果对于数$ a,b(b \neq 0) $,存在数$ q $,使得$ a = b q $,那么我们称$ b $\textbf{整除}$ a $,记作$ b \mid a $,称q是b除a的\textbf{商}.如果$ b \mid a $我们称$ b $是$ a $的一个\textbf{约数}(或一个\textbf{因子}),$ a $是$ b $的\textbf{倍数},否则记为$ b \nmid a $.
		\end{theorem} 

		\begin{prop}
			有$a,b,c$三个数:
			\begin{itemize}
				\item 如果 $ a \mid b $,$b \mid c$, 那么$a \mid c$,
				\item 如果 $ a \mid b $,$a \mid c$, 那么$a \mid bx + cy$,其中$x,y是任意整数$
			\end{itemize}
		\end{prop}
		
	\subsection{带余除法}
	
		\begin{theorem}
			对于数$ a  $和正数$ b $,存在唯一的数$ q,r $满足
			$$ 			a = b  q + r \quad  (0 \leq r < b)			$$
			我们称$ r $为$ b $除$ a $的\textbf{余数}.
		\end{theorem}

		\begin{proof}
			先证存在:
			通过改变$ bq $中$ q $的值,我们可以得到一系列的数:
			$$
			\cdots, -2b, -b, 0, b, 2b, \cdots
			$$
			从而可以得到一系列的区间:$ [ib,(i+1)b) $ ($ i $ 是整数).这些区间是整数的一个分割(并集为整数集合,任意两个的交集为空),所以$ a $必定属于其中一个,假设为$ [kb,(k+1)b) $,则可以将$ q $取成$ k $,$ r $取成$ a - kb $,则得到一组$ q, r $.
			
			下面证唯一:
			如果有两组$ q,r $:$q_1,r_1$和$q_2,r_2$(不妨假设$r_2 > r_1$),满足:
			$$ a = b q_1 + r_1  \quad (0 \leq r_1 < b)$$
			$$ a = b q_2 + r_2  \quad (0 \leq r_1 < b)$$
			做减法
			$$ r_2 -  r_1 = b (q_1 - q_2) $$
			因为$ 0 \leq r_2 - r_1 < b$,所以$r_1 = r_2$,从而$q_1 = q_2$.
		\end{proof}

	\subsection{质数}
		
		\begin{definition}
			如果一个数$ p $满足下面的性质:
			\begin{itemize}
				\item $p > 1$
				\item $ p $只有两个正因子(1和它自身)
			\end{itemize}
			那么我们称$ p $为\textbf{质数}或\textbf{素数},一个大于$ 1 $的数,如果不是质数,我们就称其为\textbf{合数},$ 1 $既不是质数,也不是合数.
		\end{definition}
		
		\begin{prop}
			任何一个合数$ a $都存在一个因子$ q $,使得$2 \leq q \leq \sqrt{a}$.
		\end{prop}
	
		\begin{proof}
			因为$ a $是合数,所以必然存在一个因子$ q $,从而$ a/q $也是$ a $的一个因子,并且满足$2 \leq q, a/q \leq a-1$,如果$ q $和$ a/q $都大于$\sqrt{a}$,则
			$q \times a/q > a$,矛盾,所以$ q $与$ a/q $中至少有一个数小于等于$\sqrt{a}$ .
		\end{proof}
		
		\begin{prop}
			质数有无穷个.
		\end{prop}
		
		\begin{proof}
			反证法.
			
			如果素数只有有限个, 那么假设他们是: $ p_1, p_2, \dots , p_s $, 那么考虑数:$ q = p_1p_2\dots p_s + 1$, 因为$ p_i \nmid q $, 所以必然存在一个素数,不在$ p_1, p_2, \dots , p_s $中. 这与我们假设矛盾.
		\end{proof}
	
	\subsection{算数基本定理}
	
		\begin{theorem}
			如果$ a $是一个大于$ 1 $的数,那么a可以被分解成一些质数的乘积,如果将质数从小到大排列,则这种分解方式是唯一的. 即任何大于$ 1 $的数可以被唯一表示成: 
			$$ a = p_1p_2p_3\dots p_s \quad ( p_1 \leq p_2 \leq p_3 \leq \dots \leq p_s  )$$
			如果把相同质数合并,那么可以被表示成: 
			$$ a = p_{1}^{\alpha_1}p_{2}^{\alpha_2}p_{3}^{\alpha_3}\dots p_{s}^{\alpha_s} \quad ( p_1 < p_2 < p_3 < \dots < p_s )$$
		\end{theorem}
		\begin{proof}
			先证明一个大于$ 1 $的数可以被分解成一些质数的乘积:
			
			归纳法:首先,$ 2,3 $显然可以写成素数的乘积.
			
			对于数假设$ 2 $到$ q-1 $都可以写成素数的乘积($q \geq 4$),现证明$q$也可以写成素数的乘积:
			
			如果$q$是质数,那么显然.
			
			如果$q$是合数,那么$q = a b$,其中$2 \leq a,b \leq q-1$,根据假设,$ a,b $可以被分解成质数的乘积,那么$q$可以被分写成质数的乘积.
			
			现在证唯一:
			
			如果有两种不同分解方法:
			$$ a = q_1 q_2 q_3 \dots q_r $$
			$$ a = p_1 p_2 p_3 \dots p_s $$
			所以:
			$$ q_1 q_2 q_3 \dots q_r = p_1 p_2 p_3 \dots p_s $$
			我们去掉两边公共的质数,从而得到(两边不会变成$ 1 $,否则就是相同的分解方案):
			$$  q_1 ' q_2 ' \dots q_{r'} '=  p_1 ' p_2 ' \dots  p_{s'}'$$
			而这是不可能的,因为$q_1 '$整除左边不整除右边.(这句话也需要证,这里我们能理解就行)
		\end{proof}
	
	\subsection{最大公约数}
	
	\begin{definition}
		对于两个数$a, b$, 如果存在数$d$, 满足$d \mid a, d \mid b$, 那么我们称$d$是$a,b$的\textbf{公因数}, 如果$a, b$不同时为$0$, 我们称其公因数中最大的称为\textbf{最大公因数}, 记作$gcd(a,b)$
	\end{definition}
	
	\begin{definition}
		对于两个非$0$数$a, b$, 如果存在数$l$, 满足$a \mid l, b \mid l $, 那么我们称$l$是$a,b$的\textbf{公倍数}, 并将其正公倍数中最小的称为\textbf{最小公倍数}, 记作$lcm(a,b)$
	\end{definition}
	
	假设$a, b$都是正的, 那么我们可以这样来理解最大公因数和最小公倍数:
	
	$$ a = p_{1}^{\alpha_1}p_{2}^{\alpha_2}p_{3}^{\alpha_3}\dots p_{s}^{\alpha_s} \quad \quad
		  b = p_{1}^{\beta_1}p_{2}^{\beta_2}p_{3}^{\beta_3}\dots p_{s}^{\beta_s} $$
	$$ gcd(a,b) = p_{1}^{min(\alpha_1,\beta_1)}p_{2}^{min(\alpha_2,\beta_2)}p_{3}^{min(\alpha_3,\beta_3)}\dots p_{s}^{min(\alpha_s,\beta_s)} $$
	$$ lcm(a,b) = p_{1}^{max(\alpha_1,\beta_1)}p_{2}^{max(\alpha_2,\beta_2)}p_{3}^{max(\alpha_3,\beta_3)}\dots p_{s}^{max(\alpha_s,\beta_s)} $$
	
	它们有一些性质:
	
	\begin{prop}
		\begin{enumerate}
				\item $gcd(a,b) = gcd(b,a)$
				\item $gcd(a,b) = gcd(a, b + ax)$, 从而有 $ gcd(a,b) = gcd(b,a \; mod \; b)$
		\end{enumerate}
	\end{prop}
	
	\begin{proof}
		第二个性质, 显然$a, b$与$a, b + ax$有相同的公因数集合,所以其最大公因数相等.
	\end{proof}
	
	\begin{theorem}
		存在数$x, y$, 使得
		$$gcd(a,b) = ax + by$$
	\end{theorem}
	\begin{proof}
		
	\end{proof}
	\begin{theorem}
		两个数的公因数整除这两个数的最大公因数,两个数的公倍数可以被这两个数的最大公倍数整除.
	\end{theorem}
	
	
	
	
	\subsection{同余}
	
	\subsection{逆元}
	
	\subsection{快速幂}
	
	\subsection{中国剩余定理}
	
	\subsection{剩余系}
	
	\subsection{欧拉函数}
	
	\subsection{欧拉定理}
	
	\subsection{费马小定理}
	
	\subsection{Wilson定理}
	
	\section{进阶内容}
	
	\subsection{素数测试}
	
	\subsection{大数质因数分解}
	
	\subsection{数论函数}

	a

\end{document} 

